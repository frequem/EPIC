\section{Marketing}\displayauthor{Florian Weinzerl}\ \\
Wenn erste Versionen dieses Produkts funktionstüchtig und einsatzbereit sind, ist allerdings noch nicht alle Arbeit getan. Der nächste Schritt ist für uns herauszufinden, wie wir unser Produkt unter die Leute bringen und es damit marktfähig machen können, und das auch umzusetzen. Da wir selbst noch keinerlei praktische Marketingerfahrung haben, nahmen wir uns einige Marketing-Werkzeuge aus dem Buch ''Marketing - Grundlagen marktorientierter Unternehmensführung\footcite{book_marketing}'' zur Hand, die sich u.a. die Abgrenzung des \textit{relevanten Marktes}, Käuferverhaltensforschung und Markenanalyse zu Nutze machten.\\

\subsection{Marktabgrenzung}
Beim Thema Marktabgrenzung scheint es verschiedene Herangehensweisen zu geben. Wir haben uns entschieden, zuerst die gängigen Kriterien und anschließend die Abgrenzung, zugeschnitten auf unser Produkt, zu behandeln.\\

\subsubsection{Kriterien zur Marktabgrenzung}
Die Kriterien, um zu bestimmen, welche Kunden in jenes Marktsegment fallen, das wir erreichen möchten, beschränken sich auf drei relativ einfache und doch wichtige Bereiche:\\

\begin{itemize}
	\item Sachlich: Welche Arten von Leistungen werden am Markt angeboten?
	\item Zeitlich: Ist der Markt zeitlich begrenzt?
	\item Räumlich: Ist der Markt lokal, regional, national oder international begrenzt?
\end{itemize}
\ \\

Die Frage um die sachlichen Kriterien würden wir so interpretieren, dass es nicht nur wichtig ist, seine direkte Konkurrenz, auf die ich später noch genauer eingehen möchte, zu betrachten, sondern auch \textit{bedürfnisorientiert} zu denken. In der Lektüre wird hierbei ein Beispiel mit einer Bohrmaschine angeführt, die nicht nur besser sein sollte als andere Bohrmaschinen, sondern auch eine praktische Alternative zu anderen Befestigungsmethoden darstellen soll. Das bedeutet übertragen auf unser Produkt, dass wir begannen, uns Gedanken zu machen, was nötig wäre, um eine volle Alternative zu unserem Produkt zu haben. Um als Kunde tatsächlich eine Alternative zu EPIC zu haben, ist es notwendig, ein Notebook als digitale Mitschreib-Gelegenheit, einen Block für Zeichnungen mit der Hand und einen Taschenrechner oder ein Mathematik-Programm für höhere Rechnungen mit Quer-Referenzen zu besitzen. Dass all dies nötig ist, ist keine Überraschung, da es genau das Grundkonzept von EPIC war, diese Dinge zu vereinen. \\
\\
Zur Frage um die zeitliche Begrenztheit ist die Antwort ganz klar ''Ja''. Unserer Meinung nach ist alles digitale oder elektronische, was für die Masse konzipiert ist, mit einem Ablaufdatum versehen. Ich denke, wir brauchen diese Meinung nicht länger zu begründen, da schon Viele feststellten, dass sich die Entwicklung der Technologie schon seit einiger Zeit im exponentiellen Wachstum zu befinden scheint. Was unsere Reaktion darauf sein sollte ist, denke ich, offensichtlich; Alle Unternehmen, die sich auf moderne Technologie spezialisieren, müssen am Puls der Zeit bleiben oder sie verlieren ihren Marktanteil. Das wird bei dieser Arbeit ähnlich verlaufen.\\
\\
Nach unserer Meinung kann man die Frage um das Räumliche auf zwei verschiedene Weisen verstehen: ''Ist der Markt im praktischen Sinne, also \underline{zur Zeit} räumlich begrenzt?'' oder ''Ist der Markt theoretisch, also mit \underline{beliebigen Kapazitäten} begrenzt?''\\
Da unsere Kapazitäten nicht beliebig wählbar sind, denke ich, macht es wenig Sinn, daran viele Gedanken zu verschwenden, aber generell kann man den Markt in etwa auf die westliche Welt einschränken, wobei möglicherweise noch Australien und Industriestaaten im Osten in Frage kämen.\\
Nun aber zum realistischen Teil, nämlich, was wir mit unserer aktuellen Marketing-Stärke erreichen können. Niederösterreich und eventuell Wien, da die Zielgruppe, wie es später noch genauer definiert wird, hauptsächlich Studierende sein sollen, sind für uns realitätsnahe Ziele. Weiters könnten auch Kunden anderer Universitätsstädten bei uns kaufen. Niederösterreich und Wien sollen jedoch vorerst unsere Hauptziele bleiben.\\

\subsubsection{Relevanter Markt}
Unter dieser Überschrift zielen wir auf eine noch detailliertere Abgrenzung ab. Mit weiteren Fragen zur Einschließung wollen wir unsere Zielgruppe konkretisieren.\\
Es stellt sich heraus, dass hier insgesamt drei für uns interessante Fragen auftreten. Andere Vorschläge für diesen Bereich sind unserem Urteil nach für Jungunternehmen nicht relevant.\\

\begin{itemize}
	\item Wie viele Nachfrager beinhaltet der Markt?
	\item Zeitlich: Wie viele Anbieter beinhaltet der Markt, und welche Anbieter gehören zu den Hauptkonkurrenten?
	\item Hat ein Unternehmen eine marktbeherrschende Stellung, sodass wettbewerbsrechtliche Vorgaben nicht mehr eingehalten werden?
\end{itemize}
\ \\
Die Nachfrager im Markt umfassen alle Studierenden, wobei der Fokus wahrscheinlich auf technischen Studienrichtungen liegt, Schüler aus höheren Schulen und möglicherweise Techniker, die viel im theoretischen Bereich zu tun haben.\\
\\
Wie bereits erwähnt sehen wir zu unserem Produkt keine direkte Konkurrenz, da zwar sowohl Taschenrechner als auch etwa ''digitale Notzblöcke'' am Markt erhältlich sind, aber nichts, was beides zusammenfasst. Aus diesem Grund möchten wir eine geteilte Konkurrenz im weiteren Sinne betrachten. Auf der Seite der Taschenrechner und Rechenprogrammen ist uns beispielsweise Mathcad wahrscheinlich am ähnlichsten, da sowohl Dokumentation als auch Speicherbarkeit und Rechenfunktionalität gegeben sind. Dafür mangelt es bei diesem Produkt zum Vergleich an Portabilität und der Möglichkeit, Freihandzuzeichnen. Als ähnliche Art der Konkurrenz könnte man auch MatLab und eventuell noch GeoGebra sehen. Auf der klassischen Taschenrechner-Seite stehen wir ganz klar TexasInstruments und zu geringerem Anteil auch Casio gegenüber. Betrachtet man hingegen rein den Aspekt der Handmitschrift, so ergibt sich als relativ moderner Konkurrent OneNote, eine Notizen-Applikation von Microsoft.\\
\\
Für eine marktbeherrschende Stellung, sodass wettbewerbsrechtliche Vorgaben nicht mehr eingehalten werden, kämen hier einerseits TexasInstruments und andererseits Microsoft in Frage. Man könnte an dieser Stelle leicht sagen, dass wir zum Glück für Microsoft keine so direkten Konkurrenten wie für beispielsweise Mathcad sind, da deren Wirtschaftsmacht die von TexasInstruments sicher bei weitem übersteigt.\\

\subsection{Aus Sicht des Kunden}
Damit wäre unser typischer Kunde nun klassifiziert. Uns ist bewusst, wen wir in etwa ansprechen, jedoch ist bis jetzt nicht geklärt wie das erfolgt. Aus diesem Grund soll in diesem Abschnitt analysiert werden, wie ein potentieller Kunde denkt und entscheidet, damit wir unser Marketing genau darauf abstimmen können.

\subsubsection{Käuferverhalten}
Im ersten Ansatz bedienten wir uns der Käuferverhaltensforschung. Diese fasst folgende wesentliche Paradigmen zusammen:\\

\begin{itemize}
	\item \tab{Wer kauft?}        \tab{$\rightarrow$ Kaufakteure, Träger der Kaufentscheidung}
	\item \tab{Was?}              \tab{$\rightarrow$ Kaufobjekte}
	\item \tab{Warum?}            \tab{$\rightarrow$ Kaufmotive}
	\item \tab{Wie?}              \tab{$\rightarrow$ Kaufentscheidungsprozess}
	\item \tab{Wie viel?}         \tab{$\rightarrow$ Kaufmenge}
	\item \tab{Wann?}             \tab{$\rightarrow$ Kaufzeitpunkt, -häufigkeit}
	\item \tab{Wo bzw. bei wem?}  \tab{$\rightarrow$ Einkaufsstätten, Lieferantenwahl}
\end{itemize}
\ \\
Da dieses Unterkapitel auf dem letzten aufbauen soll, möchte ich die Frage ''Wer kauft?'' überspringen und auf die Marktabgrenzung verweisen.\\
\\
Was gekauft wird, ist sicher auch nicht schwer zu beantworten, zu unserem Produkt gibt es keine aktuell vorgesehenen Zusatzprodukte, der Kauf beschränkt sich also einzig und allein auf das Gerät EPIC.\\
\\
Die Frage nach dem Warum ist schon interessanter. Verschiedene Unternehmen gewinnen Kunden mit verschiedenen Merkmalen. So kaufen Apple-Kunden vielleicht wegen Design oder Benutzerfreundlichkeit in der Software, und Microsoft-Kunden mögen wegen Langzeit-Support ihrer Produkte oder Kompatibilität bedingt durch die allgemeine Verbreitung der Produkte kaufen. Bei EPIC wollen wir unsere Kunden mit intuitiver Funktionalität überzeugen. Das Ziel ist es, dass Kunden keine Bedienungsanleitungen lesen und nicht mehr als einmal im Web nach einer Anleitung suchen müssen. Das sollte einer der Gründe sein, warum sich Kunden für EPIC entscheiden, gestützt von der Tatsache, dass zur Zeit, wie erwähnt, keine direkten Alternativen am Markt bestehen.\\
\\
Ich denke, der Kaufentscheidungsprozess ist für ein so junges Produkt, das noch nicht am Markt ist, schwer zu definieren, da er sich laufend ändern wird. Am Beginn wird sich sehr viel davon im Bereich der Mundpropaganda abspielen, bis später im Idealfall Werbung auf Plakate oder gar in den Medien finanziert werden könnte.\\
\\
Wie es am Technologie-Markt sicher üblich ist, wird nicht mehr als ein Exemplar pro Kopf benötigt.\\
\\
Von Kaufhäufigkeit kann hier, denke ich, nicht gesprochen werden. Je nach dem, wie sich die Technologie entwickelt, könnte es der Fortschritt alle 1-4 Jahre verlangen, ein neues Gerät anzuschaffen. Über den Kaufzeitpunkt kann womöglich auch nur spekuliert werden, aber möglicherweise zeichnet sich ein Muster ab, das im Herbst vorrangig gekauft wird, da in dieser Jahreszeit Schule und Universität beginnen.\\
\\
EPIC soll in jedem Fall im Internet erreichbar sein, womit sich die Frage nach dem Wo weitgehend klärt. Natürlich wird es auch möglich sein, Käufe über persönliches In-Kontakt-Treten abzuwickeln.\\

\subsubsection{Produktnutzen}
Die Analyse des Kaufverhaltens hat einige interessante Ergebnisse gebracht. Was allerdings noch ein wenig klarer werden könnte, ist das Kaufmotiv, weshalb wir hier den Produktnutzen in den Fokus nehmen wollen. Durch die Definition des Produktnutzens bekommen wir einen noch tieferen Einblick, warum man sich entscheiden könnte, EPIC zu kaufen.\\
\\
Eine sinnvolle Einteilung und Aufspaltung dessen ist die in \textit{Grundnutzen} und \textit{Zusatznutzen}, der sich wiederum in \textit{Erbauungsnutzen} und \textit{Geltungsnutzen} teilt. Genauer versteht man unter Grundnutzen \citebrackets{die aus den technisch-funktionalen Basiseigenschaften eines Produktes resultierende Bedürfnisbefriedigung}, die in unserem Fall eindeutig ganz simpel durch die praktische Funktionalität gegeben ist.\\
\\
Der Zusatznutzen wird eben in die \citebrackets{aus den ästhetischen Wirkungen eines Produktes} (Erbauungsnutzen) und die \citebrackets{aus den sozialen Wirkungen eines Produktes} (Geltungsnutzen) resultierende Bedürfnisbefriedigung unterteilt. Im konkreten Fall ist vielleicht mit gutem Erscheinungsbild bei einem Taschenrechner nicht viel zu holen, da er sicher nicht gekauft würde, wenn er wenig Funktionalität, dafür gutes Design aufweisen würde. Dafür kann sicher zu einem gewissen Grad der Negativ-Zufriedenheit entgegengesteuert werden; Getreu der Denkweise: ''Schlechtes Design könnte Kunden am Kaufen hindern, obwohl gutes Design den Kauf wahrscheinlich nicht fördert.'' Daraus leiten wir ab, dass die Benutzeroberfläche ansprechend gestaltet sein sollte und das Gerät eine seriöse Arbeitsweise verkörpert. Den Geltungsnutzen hingegen sehen wir bei EPIC sehr beschränkt, da sich keinerlei soziale Vorteile in einer Gruppe durch den Besitz eines Taschenrechners ergeben.\\

\subsubsection{Preisabhängige Qualitätsbeurteilung}
Als letztes Analysewerkzeug nahmen wir die Feststellung des empfundenen Kaufrisikos heran. Dieses Tool ist sehr stark mit dem Kaufentscheidungsprozess verknüpft. Entscheidet sich ein Kunde erstmalig, ein bestimmtes Produkt zu kaufen, so fand davor der Kaufentscheidungsprozess statt, mithilfe dessen er in den letzten Phasen seine präferierte Entscheidung mit Alternativen vergleicht. Als eine Art Abkürzung kann man es nun sehen, wenn das Kaufrisiko gesenkt werden kann. Beispielsweise würde ein Käufer bei einem preislich überdurchschnittlichen Mobiltelefon länger überlegen und sich fragen, ob es sich nun wirklich auszahlt, als bei einem mit durchschnittlicher Preisgestaltung. Wie man allerdings gerade am Beispiel der Mobiltelefone erkennen kann, ist es möglich dieses \textit{empfundene Kaufrisiko} auf andere Arten als durch Änderung der Preispolitik zu senken. Neben den offensichtlichen Faktoren wie Qualitätsbestreben und Zeitdruck gibt es noch viele mehr, wie in der folgenden Grafik zu sehen ist.\\
\displayimageg{img/sachsenschnitzel/book_Marketing_Kaufrisiko}{Einflussfaktoren für preisorientierte Qualitätsbeurteilung und das Kaufrisiko\footcite{book_marketing}} %Abb. 4-48 S.489 Kap. 4: 2.327

\subsection{Umsetzung}
Nach dem Einsatz all dieser Marketing-Werkzeuge ist uns nun klar, welche Kunden wir ansprechen, welche ihrer Kaufmotive wir ansprechen sollen und was ihre Entscheidung beeinflusst. Der nächste Schritt ist jetzt logischerweise, das gesammelte Wissen zu beweisen. Zusammengefasst sind unsere Ziele abgeleitet aus den Analyseergebnissen für den ersten Kundenkontakt einerseits, unsere Zielgruppe, Schüler oder Studenten nicht Technik-ferner Fachrichtungen anzusprechen, und andererseits, auf Funktionalität und Praktikabilität hinzuweisen und zu betonen. Der optimale Weg, das zu erreichen war in unseren Augen, eine Präsentation von EPIC an der Schule abzuhalten.
