\section{Hardware}
Von Anfang an war klar, dass die Hauptrecheneinheit unseres Prototypen fähig sein musste, den Rechenaufwand unserer Software problemlos zu bewältigen. Ein weiteres wichtiges Kriterium war, einen kleinen Formfarktor einhalten zu können, um zu garantieren, dass unsere Einheit in einem handlichen Gehäuße Platz findet. Beiden Punkten zugleich wird lediglich ein \textit{single-board-computer} gerecht, er vereint hohe Rechenleistung mit minimalem Platzaufkommen.\\
\\
Das \textit{Display} ist dafür konzipiert dem Benutzer ... Mittels dem \textit{resistiven Touchscreen} ist es möglich, Berechnungen einzugeben sowie Mitschriften zu tätigen. Er wird mithilfe eines \textit{Stylus} bedient.\\
\\
Da der single-board-computer und Display einen relativ hohen Energiebedarf aufweisen, musste eine Hardware-Komponente her, welche beide je nach Bedarf aus-und-einschalten kann. Bei dieser Komponente war darauf zu achten, dass diese dem Gerät so wenig Energie wie möglich entzieht um eine lange \textit{Standby-Zeit} zu garantieren. Die Komponente kann auf Knopfdruck Display und Recheneinheit getrennt von einander stromlos machen.