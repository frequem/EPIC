\subsection{3D-Druck}\displayauthor{Florian Weinzerl}\ \\
3D-Druck ist nicht gleich 3D-Druck: Der Sammelbegriff steht heute für ein ganzes Bündel von Fertigungstechniken, die nach unterschiedlichen Prinzipien funktionieren und sich jeweils nur für ganz bestimmte Materialien eignen. Ihr gemeinsamer Nenner: Alle Verfahren bauen dreidimensionale Objekte, indem sie Material in dünnen Schichten auftragen und verfestigen. Der Fachbegriff dafür ist additive Fertigung - in Abgrenzung zu subtraktiven Techniken wie Fräsen, Sägen, Bohren oder Wasserstrahlschneiden.\footcite{shit_3d_spiegel}

\subsubsection{Düsen-Technologie}
Mittels sogenanntem Fused Deposition Modeling (FDM) lassen sich nur Materialien verarbeiten, die beim Erhitzen weich und formbar werden - thermoplastische Kunststoffe wie ABS oder PLA, aber auch Modellierwachs und Schokolade. Der Druckkopf von FDM-Maschinen besteht im Kern aus einer heißen Düse, in die das feste Rohmaterial gepresst wird und sich dadurch verflüssigt. Am anderen Ende der Düse tritt es als dünner und weicher Faden aus. Damit zeichnet der Druckkopf eine Schicht des gewünschten Objekts - die äußere Kontur als einfassende Linie, Flächen werden als Schraffuren angelegt. Ist die Schicht vollendet und das Material in der gewünschten Form erstarrt, rückt der Kopf um eine Schichtdicke vom Objekt ab und zeichnet die nächste Lage.\\
\\
Die meisten 3D-Drucker, die weniger als 4000 Euro kosten, arbeiten heute mit FDM, denn die nötige Elektronik und Mechanik für eine ernstzunehmende Maschine dieses Typs ist erstaunlich simpel. Als Rohmaterial dient in der Regel Plastikdraht (Filament) von entweder 1,75 oder 3,0 Millimetern Stärke. Da dieser Draht wie bei einer Heißklebepistole im festen Zustand in die Düse gedrückt wird, sprechen manche Hersteller auch von Fused Filament Fabrication (FFF) - wahrscheinlich einfach deshalb, weil das eine griffigere Abkürzung ergibt.\\
\\
Weil der weiche Plastikdraht an der Luft nicht sofort erstarrt, müssen größere Überhänge und flache Vorsprünge am Objekt während des Drucks abgestützt werden. Bei einfachen FDM-Maschinen fügt die Software an den passenden Stellen der 3D-Vorlage geeignete Gitterstrukturen hinzu, die aus demselben Material wie das eigentliche Objekt aufgebaut werden und die man hinterher von Hand abbrechen, abknipsen oder wegschleifen muss. Gehobene FDM-Maschinen bauen die Stützstrukturen mit einem zweiten Druckkopf und aus einem anderen Material auf. Manche Stützmaterialien sind wasserlöslich oder lassen sich in einem basischen Bad auswaschen.\\
\\
\displayimageg{img/sachsenschnitzel/shit_3d_1}{Hier wurde mit Support-Material gedruckt, das anschließend entfernt werden muss\footcite{shit_3d_1}}
\noindent
Per FDM gefertigte Kunststoffteile sind belastbar und im Vergleich zu anderen additiven Techniken flott gefertigt. Allerdings weist ihre Oberfläche - trotz der heute üblichen Schichtdicken bis hinunter zu 0,1 Millimeter - oft eine deutlich sichtbare Riffelung auf. Bei manchen Objekten erinnert das an eine Holzmaserung und sieht ganz gut aus, bei vielen stört es nicht, gelegentlich wirkt es aber etwas billig.\footcite{shit_3d_spiegel}

\subsubsection{weitere Technologien}
Komplett ohne Stützmaterial kommen alle additiven Techniken aus, die ihr Rohmaterial als Pulver verarbeiten. Ein Hauch davon bildet den Stoff für jede einzelne Schicht des Modells, anschließend geht der Druckkopf darüber und verfestigt das Pulver entsprechend der Form des gewünschten Objekts. Werden Kunststoffe wie Polyamid oder Metalle wie Stahl und Titan verarbeitet, verschmilzt oder sintert ein Laser die einzelnen Körnchen punktgenau, was dann Selective Laser Melting (SLM) oder Selective Laser Sintering (SLS) heißt. Ist stattdessen ein farbiges Modell gewünscht, spritzt ein modifizierter Tintenstrahl-Druckkopf ein Gemisch aus Farbe und Bindemittel in die Pulverschicht und verklebt einzelne Körnchen. Deshalb ist speziell für diese Spielart des Pulverdrucks auch die Bezeichnung 3D-Druck (3DP) gängig, was die Begriffsverwirrung noch erhöht.\\
\\
Alles Pulver, das nicht Teil des Werkstücks wird, bleibt während des Produktionsprozesses liegen. Es stützt überhängende Teile, wird zum Schluss weggebürstet und kann für das nächste Modell wiederverwendet werden. Pulverdrucke haben oft eine raue Oberfläche, ähnlich wie feines Sandpapier. SLS und SLM produzieren robuste, elastische oder filigrane Objekte. Farbdrucker hingegen zaubern realistische Miniaturen, die allerdings etwas spröde wirken. Oft bestehen sie aus einem speziellen Polymergips und sind dadurch recht schwer.\\
\\
Die schönsten Oberflächen erzeugen Stereolithografie-Maschinen (SLA): Das Werkstück nimmt in einem Becken voller flüssigem Kunstharz Gestalt an, dessen Füllstand für jede weitere Schicht minimal erhöht wird. Die Flüssigkeit härtet unter UV-Licht punktuell aus. Entweder zeichnet ein Laser im Druckkopf die nötigen Formen in die Kunstharzoberfläche oder ein Beamer projiziert die komplette Schicht des Modells auf einmal. Stereolithografien zeigen feinste Details, sind oft aber deutlich zerbrechlicher als aus Pulver gesinterte oder gelaserte Objekte. Manche Kunstharze altern zudem sichtbar und verfärben sich dabei.\\
\\
Die Stereolithografie ist die älteste 3D-Druck-Technik und wurde bereits in den Achtzigerjahren entwickelt. Durch die 3D-Blogs geistert seit einiger Zeit der fossile Mitschnitt einer Good-Morning-America-Sendung von 1989, in der eine Stereolithografie-Maschine bei der Arbeit zu sehen ist. Das ist der wohl früheste Fernsehauftritt eines 3D-Druckers und heute noch sehenswert - nicht nur wegen der aus Haarspray in 3D betonierten Frisur der Ansagerin.\footcite{shit_3d_spiegel}

\subsubsection{Spezialtechniken}

Neben Stereolithografie, Pulverdruck und der Fertigung mit heißer Düse gibt es noch einige interessante Spezialmethoden. Manche Drucker der Objet-Serie des Herstellers Stratasys arbeiten nach dem PolyJet-Verfahren, bei dem Kunstharz tröpfchenweise gedruckt und anschließend sofort per UV-Licht gehärtet wird. Der Clou: Mehrere parallel angebrachte Düsen und Köpfe können Objekte aus verschiedenen Materialien in einem Rutsch aufbauen, beispielsweise die harte Schale einer Fernbedienung mit elastischen Tasten. Manche Drucker mischen auch Kunststoffe mit beliebig wählbaren Eigenschaften in puncto Elastizität und Farbe zusammen, was insbesondere für Prototypen bei der Produktentwicklung nützlich ist.\\
\\
Beim 3D-Druck von Edelmetall wie Gold und Silber kommt manchmal ein indirektes Verfahren zum Einsatz: Das Modell wird zunächst aus Modellierwachs in 3D gedruckt und dann konventionell im Wachsausschmelzverfahren mit verlorener Form abgegossen.\\
\\
Die irische Firma Mcor hat für beliebig farbig texturierte Drucke eine Alternative zum spröden Pulver in petto: Ihre Maschinen schichten Modelle aus ganzen Packungen gewöhnlichen Schreibpapiers auf. Für jede Schicht druckt zunächst ein konventioneller Tintenstrahlkopf einen farbigen Horizontalschnitt durch das Objekt aufs Blatt, wobei die Spezialfarbe die gesamte Dicke des Papiers durchdringt. Dann trägt die Maschine auf die Schicht darunter flüssigen Leim in Form des Objekts auf, setzt dann erst das farbig bedruckte Blatt drauf, presst es an und schneidet mit einem Messer den unverklebten Teil des Blatts entlang der Modellkontur ab. Ist das Objekt komplett aufgebaut, wird das überschüssige Papier entfernt und das Werkstück noch mal in Kunstharz getränkt. Das verleiht dem Druck eine haptisch angenehme, seidenglänzende Oberfläche, die nicht mehr im Entferntesten an Papier erinnert. In den Niederlanden kann man solche Drucke seit Kurzem bei der Büromaterialhandelskette Staples fertigen lassen. \footcite{shit_3d_spiegel}