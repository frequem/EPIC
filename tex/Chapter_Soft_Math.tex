\subsection{Das Mathematik-Paket}
\textit{Mathematik-Paket} ist der interne Name für diejenige Software Komponente, die alles in Bezug auf mathematische Vorgänge verarbeitet. Aus dem Versuch, diese Komponente in Programmcode zu fassen, erlangte ich eine wesentliche Erkenntnis. In der Mathematik sieht man Konstrukte als abstrakt, soweit sollte die mathematische Sicht der Dinge ja bekannt sein. Durch Definitionen schafft sich ein Mathematiker Freiraum, wodurch auch der Level der Abstraktion steigt. Zum Beispiel ist uns möglicherweise gar nicht mehr bewusst, dass wir beim 'mit $x$ multiplizieren', tatsächlich $x$ Mal zusammenzählen. Beim 'hoch $x$ rechnen' wird eigentlich $x$ Mal multipliziert, also wie oft addiert? Natürlich stellen die Grundrechenarten programmtechnisch kein Problem dar, jedoch ist bereits diese einfache Frage ohne mehr Information nicht zu beantworten, was zeigen soll, wie tief wir schon drin stecken, ohne jemals angefangen zu haben.\\\\
