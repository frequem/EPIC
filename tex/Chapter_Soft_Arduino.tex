\subsection{Arduino und C}

Um das Ein- und Ausschalten der Hardware-Komponenten möglich zu machen wurde der verwendete Arduino Mikrocontroller mithilfe der Arduino IDE programmiert. Der Arduino wird mit der Programmiersprache C programmiert, da ich diese Programmiersprache bereits beherrscht habe, ist mir die Programmierung leicht gefallen.\\
\\
Es sollten folgende Spezifikationen erfüllt werden:
\begin{itemize}
	\item Wenn der Knopf gedrückt und innerhalb von 3 Sekunden losgelassen wird soll, solange der Raspberry Pi eingeschaltet ist, der Displaystrom getoggelt, also je nach Zustand aus- oder eingeschaltet werden.
	\item Wenn der Knopf gedrückt und nach 3-sekündigem Halten losgelassen wird soll, solange der Raspberry Pi eingeschaltet ist, dem Raspberry Pi ein Signal zum Herunterfahren geschickt werden. Sobald dieser Shutdown-Prozess beendet ist sollen Raspberry und Display vom Strom getrennt werden.
	\item Wenn der Knopf gedrückt und nach 3-sekündigem Halten losgelassen wird soll, solange der Raspberry Pi ausgeschaltet ist, der Strom zum Raspberry Pi und zum Display frei gegeben d.h. eingeschaltet werden.
\end{itemize}
\ \\
Um im Standby-Modus (wenn Display und Raspberry Pi ausgeschaltet sind) möglichst stromsparend zu arbeiten, wurde die LowPower-Library \footcite{lowpower_lib} verwendet und mit Interrupts programmiert, dies macht es möglich den Mikrocontroller in einen Schlafmodus zu versetzen, so hat dieser einen sehr geringen Arbeitsstrom.
\\
Dies wurde folgendermaßen realisiert:
\displayownimageg{img/frequem/arduino_loop}{Arduino Event Loop}{Michael Friesenhengst}