\section{Bedienungsanleitung}

\subsection{Hardware-Bedienung}

\subsubsection{Powerbutton}

Der Powerbutton wird verwendet, um EPIC einzuschalten, den Standbyzustand zu beenden, ihn zu starten oder um EPIC auszuschalten. Beim Versetzen in den Ruhezustand wird der Bildschirm ausgeschaltet, was Strom spart und das Berühren unbeabsichtigter Funktionen verhindert.\\
\\
Die Standbytaste befindet sich seitlich rechts.


\paragraph{EPIC einschalten:}
Halten Sie den Powerbutton für mind. 3 Sekunden gedrückt. Nach anschließendem Loslassen wird EPIC eingeschaltet.

\paragraph{EPIC ausschalten:}
Halten Sie den Powerbutton für mind. 3 Sekunden gedrückt. Nach anschließendem Loslassen wird EPiC heruntergefahren und ausgeschalten.

\paragraph{Standbymodus starten:}
Drücken Sie kurz auf den Powerbutton und lassen diesen umgehend wieder los. Ist der Bildschirm von EPIC abgeschaltet, so ist der Standbymodus aktiv.

\paragraph{Standbymodus beenden:}
Drücken Sie kurz auf den Powerbutton und lassen diesen umgehend wieder los. Das Gerät ist nun wieder einsatzfähig.

\subsubsection{Mikro-USB-Anschluss}

Verbinden sie ein passendes USB-Kabel mit dem Mikro-USB-Anschluss um EPIC aufzuladen. 

\subsubsection{USB-Anschluss}

Der USB-Anschluss kann wie ein regulärer USB-Port eines PCs genutzt werden. Es können unter anderem USB-Memory-Sticks, Tastaturen oder Mäuse verbunden und genutzt werden.

\subsubsection{Touchscreen}

Der Touchscreen wird mithilfe eines Stiftes bedient, dieser sollte eine Plastik- oder Nylonspitze haben um problemlos auf dem Touchscreen zu gleiten. Um gute Ergebnisse zu erzielen wird empfohlen ca. 1,5x so fest aufzudrücken, wie man es mit einem Kugelschreiber gewöhnt ist. 

\subsection{Software-Bedienung}

\subsubsection{Spritepanel}
Das Spritepanel ist der große weiße Bereich in der Mitte des Fensters. Auf ihm werden alle Aktionen durchgeführt, es werden Berechnungen durchgeführt, Skizzen angelegt oder Notizen verfasst. Einzelne Objekte, welche auf dem Spritepanel angezeigt werden können, wie Text oder Zeichnungen werden als Sprites bezeichnet.

\subsubsection{Verwenden der On-Screen-Tastatur}

\subsubsection{Menubar}

Die Menubar ist der Menübereich über dem Spritepanel, er ist zur Konfiguration und zum Starten verschiedener Operationen gedacht. Um eine Operation zu starten oder eine Einstellung zu treffen, wird auf die jeweiligen Buttons, die nach Kategorien aufgelistet sind, geklickt. In manchen Fällen, sind zu viele Operationen (Menuitems) innerhalb einer Kategorie verfügbar, es wird unten ein Button zur Erweiterung des Menüs angezeigt, wird auf diesen Button geklickt, so öffnet sich ein neues Fenster auf dem die zusätzlichen Auswahlmöglichkeiten angezeigt werden. Sie können wie gewohnt mittels Klick auf den Touchscreen verwendet werden.

\subsubsection{Modes}

Verschiedene Modes werden auf der Menubar ausgewählt.

\paragraph{Draw Mode}
\paragraph{Text Mode}
\paragraph{Math Mode}
\paragraph{Selection Mode}
\paragraph{Move Mode}

\subsubsection{Änderung der Farbe}
\subsubsection{Änderung der Strichstärke}
\subsubsection{Löschen von Sprites}
\subsubsection{Rückgängigmachen einer Action}
