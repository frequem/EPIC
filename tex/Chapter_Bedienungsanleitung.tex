\section{Bedienungsanleitung}\displayauthor{Michael Friesenhengst}
\subsection{Hardware-Bedienung}

\subsubsection{Powerbutton}

Der Powerbutton wird verwendet, um EPIC einzuschalten, den Standbyzustand zu beenden, ihn zu starten oder um EPIC auszuschalten. Beim Versetzen in den Ruhezustand wird der Bildschirm ausgeschaltet, was Strom spart und das Berühren unbeabsichtigter Funktionen verhindert.\\
\\
Die Standbytaste befindet sich seitlich rechts.


\paragraph{EPIC einschalten:}
Halten Sie den Powerbutton für mind. 3 Sekunden gedrückt. Nach anschließendem Loslassen wird EPIC eingeschaltet.

\paragraph{EPIC ausschalten:}
Halten Sie den Powerbutton für mind. 3 Sekunden gedrückt. Nach anschließendem Loslassen wird EPIC heruntergefahren und ausgeschalten.

\paragraph{Standbymodus starten:}
Drücken Sie kurz auf den Powerbutton und lassen Sie diesen umgehend wieder los. Ist der Bildschirm von EPIC abgeschaltet, so ist der Standbymodus aktiv.

\paragraph{Standbymodus beenden:}
Drücken Sie kurz auf den Powerbutton und lassen Sie diesen umgehend wieder los. Das Gerät ist nun wieder einsatzfähig.

\subsubsection{Mikro-USB-Anschluss}

Verbinden sie ein passendes USB-Kabel mit dem Mikro-USB-Anschluss um EPIC aufzuladen. 

\subsubsection{USB-Anschluss}

Der USB-Anschluss kann wie ein regulärer USB-Port eines PCs genutzt werden. Es können unter anderem USB-Memory-Sticks, Tastaturen oder Mäuse verbunden und genutzt werden.

\subsubsection{Touchscreen}

Der Touchscreen wird mithilfe eines Stiftes bedient, dieser sollte eine Plastik- oder Nylonspitze haben, um problemlos auf dem Touchscreen zu gleiten. Um gute Ergebnisse zu erzielen, wird empfohlen ca. 1,5x so fest aufzudrücken, wie man es mit einem Kugelschreiber gewöhnt ist. 

\subsection{Software-Bedienung}

\subsubsection{Spritepanel}
Das Spritepanel ist der große weiße Bereich in der Mitte des Fensters. Darauf werden alle Aktionen durchgeführt, es werden Berechnungen getätigt, Skizzen angelegt oder Notizen verfasst. Einzelne Objekte, welche auf dem Spritepanel angezeigt werden können wie Text oder Zeichnungen, werden als Sprites bezeichnet.

\subsubsection{Verwenden der On-Screen-Tastatur}

Die On-Screen-Tastatur ist nach dem Hochfahren standardmäßig maximiert angezeigt. Um die Tastatur zu minimieren, wird auf den standardmäßig rechts unten am Bildschirm angezeigten Keyboard-Icon geklickt, dieser wird ebenfalls verwendet um die Tastatur zu maximieren. Der Icon und die Tastatur können durch Halten und Bewegen des Stiftes am Bildschirm verschoben werden.


\subsubsection{Menubar}

Die Menubar ist der Menübereich über dem Spritepanel. Er ist zur Konfiguration und zum Starten verschiedener Operationen gedacht. Um eine Operation zu starten oder eine Einstellung zu treffen, wird auf die jeweiligen Buttons, die nach Kategorien aufgelistet sind, geklickt. In manchen Fällen, sind zu viele Operationen (Menuitems) innerhalb einer Kategorie verfügbar, in diesem Fall wird unten ein Button zur Erweiterung des Menüs angezeigt. Wird auf diesen Button geklickt, so öffnet sich ein neues Fenster, auf dem die zusätzlichen Auswahlmöglichkeiten angezeigt werden. Diese können wie gewohnt mittels Klick auf den Touchscreen verwendet werden.

\subsubsection{Modes}

Verschiedene Modes werden auf der Menubar ausgewählt.

\paragraph{Draw Mode:}
Im Draw-Mode können Zeichnungen, Skizzen und handschriftliche Notizen angelegt werden. Wählen Sie dazu den Draw Mode in der Menubar aus, anschließend können durch Streichen des Stiftes über den Bildschirm auf dem Spritepanel Zeichnungen angelegt werden. Wird der Stift vom Spritepanel abgesetzt, so wird beim nächsten Streichen ein neues Sprite angelegt, mit dem unabhängig vom alten gearbeitet werden kann.

\paragraph{Text Mode:}

Im Text-Mode können mithilfe der On-Screen Tastatur Texte verfasst werden. Nach dem Auswählen des Text-Modes in der Menubar kann auf eine beliebige Stelle im Spritepanel geklickt werden. An der gewählten Stelle erscheint ein blauer Cursor, mithilfe der Tastatur kann nun ein Text verfasst werden.\\
\\
Im Text-Mode können mithilfe des Cursors auch mehrere Buchstaben ausgewählt werden um diese beispielsweise zu ersetzten oder zu löschen. Dazu können innerhalb eines Text-Sprites Textstellen mit dem Stift markiert werden, wobei ausgewählte Buchstaben blau hinterlegt werden. Markierte Buchstaben können durch Schreiben auf der Tastatur ersetzt bzw. mithilfe der Backspace-Taste gelöscht werden.

\paragraph{Math Mode:}

Im Math-Mode können mithilfe der On-Screen-Tastatur Berechnungen durchgeführt werden. Nach dem Auswählen des Math-Modes in der Menubar kann auf eine beliebige Stelle im Spritepanel geklickt werden. An der gewählten Stelle erscheint ein blauer Cursor, mithilfe der Tastatur können nun Berechnungen durchgeführt werden.\\
\\
Die Auswahl mehrerer Terme erfolgt wie im Text-Mode.
\\
Folgende Operationen sind möglich:

\begin{itemize}
	\item Konstante - Ganz- oder Gleitkommazahl (z.B. 9.4)
	\item Variable - Name einer Variable (z.B. x)
	\item + - Addition
	\item - - Subtraktion
	\item * - Multiplikation
	\item / - Division
	\item (...) - Klammer
	\item = - wertet eine Berechung aus
\end{itemize}

\paragraph{Selection Mode:}

Im Selection Mode können mehrere Sprites ausgewählt werden, um mit diesen anschließend eine Operation durchzuführen. Entweder durch einen Klick auf ein Sprite oder durch Ziehen eines Rechtecks über mehrere Sprites können diese ausgewählt werden. Selektierte Sprites werden mit einem blau-strichlierten Rahmen gekennzeichnet.

\paragraph{Move Mode:}

Im Move-Mode können alle selektierten Sprites auf dem Bildschirm verschoben werden. Sie können entweder zuvor mithilfe des Selection-Mode oder im Move-Mode durch einfachen Klick auf ein Sprite ausgewählt werden. Streichen sie von der Ursprungsposition zur Zielposition um die Sprites entsprechend zu verschieben.

\subsubsection{Änderung der Farbe}

Zum Ändern der Strich- und Textfarbe gibt es in der Menubar ein entsprechendes Item. Drücken Sie auf ''Set Color'' und wählen sie in dem neu geöffneten Fenster durch Streichen über dem Touchscreen eine neue Farbe aus. Diese kann durch einen Klick auf ''OK'' bestätigt werden. Die aktuell ausgewählte Farbe wird in der Menubar neben dem ''Set Color''-Menuitem angezeigt.

\subsubsection{Änderung der Strichstärke}

Zum Ändern der Strichstärke drücken sie auf den Stroke-Dropdown in der Menubar (Strich über das gesamte Item mit einem Pfeil nach unten). Aus der erscheinenden Liste können Sie zwischen fünf verschiedenen Strichstärken wählen. Die Strichstärke wird im Draw-Mode berücksichtigt.

\subsubsection{Löschen von Sprites}

Um ein oder mehrere Sprites zu löschen, müssen diese vorher ausgewählt werden. Wählen Sie mithilfe des Selection-Mode die zu löschenden Sprites aus und drücken Sie anschließend auf das ''Delete''-Menuitem in der Menubar.

\subsubsection{Rückgängigmachen einer Action}

Actions wie das Ändern der Farbe, der Strichstärke, das Löschen von Sprites, etc. können rückgängig gemacht werden. Durch einen Klick auf ''Back'' in der Menubar wird der Zustand auf jenen vor der letzten Action zurückgestellt. Falls man in der Historie zu weit zurück geht kann durch Klicken auf ''Forward'' eine Action wiederhergestellt werden.

\subsubsection{Speichern einer Datei}

Um den aktuellen Zustand des Spritepanels zu Speichern und ihn später wiederherzustellen, kann durch einen Klick auf ''Save'' bzw. ''Save as'' in der Menubar eine Datei mit den aktuellen Inhalten des Spritepanels erstellt werden.\\
\\
Bei erstmaliger Speicherung und einem Klick auf ''Save'' bzw. durch einen Klick auf ''Save as'' muss der Benutzer einen Dateipfad angeben unter dem die aktuelle Datei gespeichert wird. Bei wiederholter Speicherung in die selbe Datei wird dieser Dialog automatisch übersprungen und der letzte Pfad verwendet.

\subsubsection{Laden einer Datei}

Um den Zustand einer bereits gespeicherten Datei wiederherzustellen, kann durch einen Klick auf ''Load'' eine Datei ausgewählt werden. Nach Auswählen dieser Datei wird der entsprechende Zustand auf dem Spritepanel wiederhergestellt.