\subsection{GUI - Das Graphical User Interface}

Das GUI ist der Teil unserer Software welcher dem Benutzer eine Oberfläche bietet, auf der er alle seine Berechnungen, Notizen, etc. durchführen kann. Wir überlegten uns, dass dem Benutzer etwas zur Verfügung stehen sollte, auf dem er alles tun kann was die Software bietet. Wir wollten dafür ein zentrales Fenster, da dadurch die Bedienung wesentlich erleichtert wird und auch verschiedene Dinge wie Berechnungen oder Notizen nebeneinander, in einer dem Benutzer überlassenen Anordnung, angezeigt werden kann. Dieses Zentrale Fenster bezeichneten wir als das \textit{Spritepanel}.\\
\\
Alle Objekte, die auf dem Spritepanel angezeigt werden können, werden in unserer Struktur als \textit{Sprites} bezeichnet. Dies ist auch in der Objektstruktur das gemeinsame Glied, auf dem alle zeichenbaren Objekte, wie \textit{Terms} oder \textit{Drawings} aufsetzen. Im Programmcode wurde das Sprite als Interface ausgeführt.\\
\\
Um auf dem Spritepanel verschiedene Dinge, wie zeichnen, schreiben, ... durchführen zu können muss es eine Möglichkeit geben, zwischen diesen Modi schnell und effizient hin- und herzuschalten, auch um andere Operationen wie das Wechseln der Schriftfarbe oder der Strichstärke durchführen zu können musste es eine einfache Lösung geben. Wir entschieden uns dafür, auf dem oberen Rand des Bildschirmes eine Menüleiste zu erstellen, auf der es dem Benutzer möglich ist alle das Spritepanel betreffenden Optionen zu tätigen.\\
\\
Alle Optionen, welche sinntechnisch zusammen gehören wurden auch zusammen in Untermenüs gegliedert. Jedes dieser Untermenüs hat dann zu jeder Optionen jeweils ein Item, mit dem diese Option steuerbar ist. Diese Items können nach einem Touch auf den Touchscreen verschiedene Reaktionen aufweisen. So ist es möglich dass ein Item nach einem Touch ein neues Fenster spawnt, auf dem der Benutzer seine Einstellungen treffen kann, auf einem anderem Item wiederum kann man seine Einstellung direkt treffen.