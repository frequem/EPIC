\documentclass[a4paper, 12pt]{scrartcl}

% !TeX spellcheck = de_AT

\usepackage[utf8]{inputenc}
\usepackage[a4paper, total={7in, 10in}]{geometry}
\usepackage[ngerman]{babel}
\usepackage[T1]{fontenc}
\usepackage{amsmath}
\usepackage{amsfonts}
\usepackage{hyperref}
\usepackage{graphicx}
\usepackage{setspace}
\usepackage[normalem]{ulem}
\usepackage{fancyhdr}
\usepackage{datetime}
\usepackage[style=verbose]{biblatex}
\usepackage{systeme} %for math sys o. equ.
\usepackage{float} %needed to display images
\usepackage{ifthen,xcolor} %somehow needed for tabs
\usepackage{pdfpages} %for importing pdfs
\usepackage{longtable} %for long tables
\usepackage{tabu}%for long tables
\usepackage{footnote}

\newlength{\tabcont} %something with tabs


%systeme config
\sysdelim..
\syslineskipcoeff{1.2}\setlength{\tabskip}{3pt}

\newcommand{\displayauthor}[1]{
	{\small\textit{verfasst von: {#1}}}
}

\newcommand{\blank}[1]{\hspace*{#1}\linebreak[0]}
%user defined commands
\newcommand{\tab}[1]{
	% weils sonst anscheinend keinen tab gibt in latex
	% verwenden:
	% \tab{1} \tab{2}
	% \tab{3} \tab{4}
	% gibt:
	% 1    2
	% 3    4
	\settowidth{\tabcont}{#1}
	\ifthenelse{\lengthtest{\tabcont < .25\linewidth}}
	{\makebox[.25\linewidth][l]{#1}\ignorespaces}
	{\makebox[.5\linewidth][l]{#1}\ignorespaces}
}

\newcommand{\displaycode}[1] {
	% fallst mal versuchst, code mit syntax-highlighting darzustellen:
	% geht anscheinend nicht in makros, da diese code-interpreter
	% angeblich über sich gegenseitig drüberstolpern, habs weder mit
	% 'listings' noch mit 'minted' zum laufen kriegt...
	% is a ziemlicher müll, mach lieber screenshots und füg die ein
	{\fontfamily{qcr}\selectfont #1}
}

\newcommand{\displayownimageg}[3]{
	% wieder gschissen, aber hängt einen Verweis
	% auf den Bildinhaber an
	\displayownimage{#1}{#2}{#3}{0.9}
}

\newcommand{\displayimageg}[2]{
	% oida in latex kamma 'commands' mit untersch. parameteranzahl
	% nicht gleich nennen... ich nenn das jetzt _g für gschissen
	% 0.9 is halt meine standardeinstellung, schaut nicht so deppat
	% aus wie 1, aber ma sieht noch was
	\displayimage{#1}{#2}{#2}{0.9}
}

\newcommand{\displayownimage}[4]{
	\displayimage{#1}{#2}{#2 (Quelle: #3)}{#4}
}

\newcommand{\displayimage}[4]{
	\begin{figure}[H]
	\begin{center}
		\includegraphics[width=#4\textwidth]{#1}
		\caption[#3]{#2} % könnt nützlich werden für das was der schaupp gsagt hat
	\end{center}
	\end{figure}
}

\newcommand{\citebrackets}[1]{
	% wennst iwas zitierst bitte die funktion verwenden. wieder ausm
	% gleichen grund: zentralisieren, wenn iwas net passt, hier ändern
	<<#1>>
}

\newcommand{\makeowntitle}{
	\begin{titlepage}
		\centering
		{\scshape\Large {HTBL Hollabrunn}\\}
		{\scshape\large {Höhere Lehranstalt für Wirtschaftsingenieure}\\}
		{\scshape\normalsize Ausbildungsschwerpunkt Betriebsinformatik\\}
		
		\vspace{4.5cm}
		
		{\huge \textbf{EPIC}}\\
		\vspace{1cm}
		{\LARGE \textbf{Electronic Programmable Intelligent Calculator}}\\
		\vspace{1.5cm}
		%\vfill
		{\Large Michael Friesenhengst, Florian Weinzerl}\\
		\vspace{0.5cm}
		{\Large Betreuer: Ing. Leopold Mayer MBA StR}\\
		\vspace{1.5cm}
		{\large \today}
	\end{titlepage}
}

%add refs
\addbibresource{ref/sachsenschnitzel_references.bib}
\addbibresource{ref/frequem_references.bib}

%\addcontentsline{toc}{section}{Literatur} wemma die Literatur im toc haben wollen, müsst aber noch getweak't werden

\newdateformat{myformat}{\THEDAY{ten }\monthname[\THEMONTH], \THEYEAR}
\pagestyle{fancy}
\fancyhf{}
\rhead{Florian Weinzerl, Michael Friesenhengst}
\lhead{\leftmark}

\DeclareGraphicsExtensions{.pdf,.png,.jpg,.jpeg}
\onehalfspacing

\begin{document}

\makeowntitle
\newpage
{
	\centering
	{\scshape\Large {HTBL Hollabrunn}\\}
	{\scshape\large {Höhere Lehranstalt für Wirtschaftsingenieure}\\}
	{\scshape\normalsize Ausbildungsschwerpunkt Betriebsinformatik\\}
}
\vspace{1.5cm}
\section*{Eidesstattliche Erklärung}
\vspace{0.5cm}
Wir erklären an Eides statt, dass wir die vorliegende Diplomarbeit selbstständig und ohne fremde Hilfe verfasst, andere als die angegebenen Quellen und Hilfsmittel nicht benutzt und die den benutzten Quellen wortwörtlich und inhaltlich entnommenen Stellen als solche erkenntlich gemacht haben.\\
\\
Hollabrunn, am 04.04.2017\\
\vspace{1cm}\\
\textbf{Florian Weinzerl}\\
\\
\rule{8cm}{0.15mm}\\
\\
\textbf{Michael Friesenhengst}\\
\\
\rule{8cm}{0.15mm}
\includepdf[pages={1,2}]{ext/HTL_RDP_Dokumentation_DA_DE_A4.pdf}
\includepdf[pages={1,2}]{ext/HTL_RDP_Dokumentation_DA_EN_A4.pdf}

\tableofcontents
\newpage

\setcounter{page}{1}
\cfoot{\thepage}

% !TeX spellcheck = de_AT
\section{Einleitung}
Bereits als wir uns erstmalig mit dem Thema, eine Diplomarbeit zu schreiben, auseinandersetzten, war es eindeutig, dass sie nicht bloß sinnhaft und innovativ, sondern darüber hinaus noch eine Herausforderung für uns sein müsste. Ein solches Projekt zu finden, stellte sich als nicht einfach heraus, bis schlussendlich ein Problem aus dem Schulunterricht auftrat. Das Wort "Problem" soll hier weniger für eine Störung stehen, wie es in unserem Sprachgebrauch oft verwendet wird, viel mehr ist damit "Problemstellung" gemeint.\\\\
In unseren jeweiligen Schulkarrieren wurde denkbar auf Papier begonnen und allmählich würde ein Fach nach dem anderen auf  digitale Mitschrift durch Laptop oder ähnliches umsteigen. Heute und vermutlich noch sehr lange Zeit besteht dies aus einer Mischform: Wer bestmöglich im Unterricht dabei sein will, sollte Texte notieren, möglicherweise auf Dateien aus dem Internet zugreifen, Freihandzeichnungen von der Tafel auf die eigenen Zettel übertragen und die Lösung des einen oder anderen Gleichungssystems nachvollziehbar dokumentieren können und wenn möglich alles so speichern respektive hinterlegen, dass es einfach wieder auffindbar ist. Unserer Meinung nach, sind das zwar Aufgaben, die bewältigbar sind, aber wenn es möglich wäre, all diese Aufgabenbereiche in ein Gerät zu kombinieren, das demnach im Unterricht, in Vorlesungen, aber auch in manchen technischen Berufen genutzt werden kann, so ist das Ergebnis nicht nur eine Entlastung, sondern schließt womöglich auch eine Marktlücke.\\\\
Und genau das haben wir uns zur Aufgabe gemacht. Wenn Sie als Leser eine kurze Werbung erlauben, hier ist die fertig verfasste Aufgabenstellung an uns selbst während der Arbeit:\\\\
\textit{EPIC ist ein fortschrittlicher Taschenrechner in Form eines Tablets, das mit einem Stylus bedient werden kann. EPIC kann sowohl Berechnungen durchführen als auch Funktionsgraphen erstellen sowie Gleichungen lösen. Er soll die Mitschrift händisch sowie am Comptuer ersetzen und somit alles in einem Gerät vereinen. Das gesamte System befindet sich auf einer austauschbaren SD-Karte, wodurch einfache Software-Updates bzw. Resets durchgeführt werden können. Der langlebige Akku garantiert Arbeiten ohne Unterbrechung.}\\\\
Dinge, die uns außerdem wichtig sind, umfassen unter anderem Modularität und das korrekte Marketing, was wir uns anschließend auch zur Aufgabe machten.

\subsection{Gedanken zur Umsetzung}
%Wollma bei der Umsetzung dazu schreiben, wer was gmacht hat? oder iwo anders?
Nun stellt sich also die Frage: Wie wollen wir all das bewerkstelligen? Die hardwaretechnische Einschränkung auf ein Tablet erleichtert die Entscheidung schon um Einiges. Im Hinblick darauf fassten wir den Entschluss, ein \textbf{...Friesi pls...}\\\\
Bezüglich der Software konnte wohl nicht viel aus fremden Komponenten verwendet werden, vor allem, da das dem Sinn dieser Arbeit widersprechen würde. Wir waren also selbst dafür zuständig und das bedeutete, sowohl die Benutzer-Oberfläche als auch die Schreib- und Mathematik-Einheit waren so zu realisieren, dass auch nach Abschluss und Veröffentlichung Entwickler auf dem Code aufbauen können. Das hat zusätzlich den taktischen Vorteil, dass Benutzer bei Bedarf auch schnell eigene Funktionen hinzufügen und diese dann für andere zugänglich machen können. Diese Funktionalitäten sind zwar von den üblichen Taschenrechnern bekannt, allerdings nicht in einer plattformunabhängigen Sprache, die darüber hinaus objektorientiertes Programmieren ermöglicht, wie Java.
%eigentlich iss ja net ganz teil der diplomarbeit, aber schau ma obs ihm auffallt?:
Potenzielle Kunden hätten also den Vorteil, dass sie selbst darüber verfügen können, welche Software-Pakete sie neben den Grundfunktionalitäten integriert haben möchten. Das bezieht sich auch darauf, dass sie vom Programmiertalent Anderer profitieren können, falls eine eher ausgefallene oder fachspezifische Funktion per Softwaremodul von einer Privatperson ins Internet hoch geladen wurde.\\\\
Betreffend der Oberfläche ist sicher ein Menü am oberen Fensterrand sinnvoll. Dieses sollte Menüpunkte in Kategorien einteilen, und dem Benutzer die häufigst verwendeten Funktionen schnell zugreifbar machen. Dies sind möglicherweise keine überraschenden Festlegungen, aber zu irgendeinem Zeitpunkt müssen Design-Entscheidungen getroffen werden. Außerdem weist nicht jedes Menü diese Flexibilität auf, ob sie sich bewährt wird man sehen, und diese Eigenschaft kann eventuell noch nachjustiert werden.
Darüber, was die Mathematik-Einheit zu bieten haben sollte, lässt sich wohl kaum streiten. Die Vorgaben dafür beschränken sich, bis auf, dass sie funktionsfähig sein soll, auf Null.\\\\
Dafür ist das Marketing eine äußert interessante, um nicht zu sagen eine schwungvolle Angelegenheit. Dabei wollen wir uns in unsere potenziellen Kunden hinein versetzen und erforschen, warum sie unser Produkt kaufen, wie der Entscheidungsprozess stattfindet und in wie fern wir uns das für unseren Imageaufbau und Vermarktung des Produkts zu Nutzen machen können.
\section{Bedienungsanleitung}

\subsection{Hardware-Bedienung}

\subsubsection{Powerbutton}

Der Powerbutton wird verwendet, um EPIC einzuschalten, den Standbyzustand zu beenden, ihn zu starten oder um EPIC auszuschalten. Beim Versetzen in den Ruhezustand wird der Bildschirm ausgeschaltet, was Strom spart und das Berühren unbeabsichtigter Funktionen verhindert.\\
\\
Die Standbytaste befindet sich seitlich rechts.


\paragraph{EPIC einschalten:}
Halten Sie den Powerbutton für mind. 3 Sekunden gedrückt. Nach anschließendem Loslassen wird EPIC eingeschaltet.

\paragraph{EPIC ausschalten:}
Halten Sie den Powerbutton für mind. 3 Sekunden gedrückt. Nach anschließendem Loslassen wird EPIC heruntergefahren und ausgeschalten.

\paragraph{Standbymodus starten:}
Drücken Sie kurz auf den Powerbutton und lassen diesen umgehend wieder los. Ist der Bildschirm von EPIC abgeschaltet, so ist der Standbymodus aktiv.

\paragraph{Standbymodus beenden:}
Drücken Sie kurz auf den Powerbutton und lassen diesen umgehend wieder los. Das Gerät ist nun wieder einsatzfähig.

\subsubsection{Mikro-USB-Anschluss}

Verbinden sie ein passendes USB-Kabel mit dem Mikro-USB-Anschluss um EPIC aufzuladen. 

\subsubsection{USB-Anschluss}

Der USB-Anschluss kann wie ein regulärer USB-Port eines PCs genutzt werden. Es können unter anderem USB-Memory-Sticks, Tastaturen oder Mäuse verbunden und genutzt werden.

\subsubsection{Touchscreen}

Der Touchscreen wird mithilfe eines Stiftes bedient, dieser sollte eine Plastik- oder Nylonspitze haben um problemlos auf dem Touchscreen zu gleiten. Um gute Ergebnisse zu erzielen wird empfohlen ca. 1,5x so fest aufzudrücken, wie man es mit einem Kugelschreiber gewöhnt ist. 

\subsection{Software-Bedienung}

\subsubsection{Spritepanel}
Das Spritepanel ist der große weiße Bereich in der Mitte des Fensters. Auf ihm werden alle Aktionen durchgeführt, es werden Berechnungen durchgeführt, Skizzen angelegt oder Notizen verfasst. Einzelne Objekte, welche auf dem Spritepanel angezeigt werden können, wie Text oder Zeichnungen werden als Sprites bezeichnet.

\subsubsection{Verwenden der On-Screen-Tastatur}

Die On-Screen-Tastatur ist nach dem Hochfahren standardmäßig maximiert angezeigt. Um die Tastatur zu minimieren wird auf den standardmäßig rechts unten am Bildschirm angezeigtes Keyboard-Icon geklickt, dieser wird ebenfalls verwendet um die Tastatur wieder zu maximieren. Das Icon und die Tastatur können durch Halten und Bewegen des Stiftes am Bildschirm verschoben werden.


\subsubsection{Menubar}

Die Menubar ist der Menübereich über dem Spritepanel, er ist zur Konfiguration und zum Starten verschiedener Operationen gedacht. Um eine Operation zu starten oder eine Einstellung zu treffen, wird auf die jeweiligen Buttons, die nach Kategorien aufgelistet sind, geklickt. In manchen Fällen, sind zu viele Operationen (Menuitems) innerhalb einer Kategorie verfügbar, es wird unten ein Button zur Erweiterung des Menüs angezeigt, wird auf diesen Button geklickt, so öffnet sich ein neues Fenster auf dem die zusätzlichen Auswahlmöglichkeiten angezeigt werden. Sie können wie gewohnt mittels Klick auf den Touchscreen verwendet werden.

\subsubsection{Modes}

Verschiedene Modes werden auf der Menubar ausgewählt.

\paragraph{Draw Mode:}
Im Draw-Mode können Zeichnungen, Skizzen und handschriftliche Notizen angelegt werden. Wählen Sie dazu den Draw Mode in der Menubar aus, anschließend können durch Streichen des Stiftes über dem Bildschirm auf dem Spritepanel Zeichnungen angelegt werden. Wird der Stift vom Spritepanel abgesetzt, so wird beim nächsten Streichen ein neues Sprite angelegt, mit dem unabhängig vom alten gearbeitet werden kann.

\paragraph{Text Mode:}

Im Text-Mode können mithilfe der On-Screen Tastatur Texte verfasst werden. Nach dem Auswählen des Text-Modes in der Menubar kann auf eine beliebige Stelle im Spritepanel geklickt werden. An der gewählten Stelle erscheint ein blauer Cursor, mithilfe der Tastatur kann nun ein Text verfasst werden.\\
\\
Im Text-Mode können mithilfe des Cursors auch mehrere Buchstabe ausgewählt werden um diese beispielsweise zu ersetzten oder zu löschen. Dazu kann innerhalb eines Text-Sprites mit dem Stift über die verschiedenen Buchstaben gewischt werden, ausgewählt Buchstaben werden blau hinterlegt dargestellt. Ausgewählte Buchstabe können durch Schreiben auf der Tastatur ersetzt bzw. mithilfe der Backspace-Taste gelöscht werden.

\paragraph{Math Mode:}

Im Math-Mode können mithilfe der On-Screen-Tastatur Berechnungen durchgeführt werden. Nach dem Auswählen des Text-Modes in der Menubar kann auf eine beliebige Stelle im Spritepanel geklickt werden. An der gewählten Stelle erscheint ein blauer Cursor, mithilfe der Tastatur können nun Berechnungen durchgeführt werden.\\
\\
Die Auswahl mehrerer Terme erfolgt wie im Text-Mode.
\\
Folgende Operationen sind möglich:

\itemize{
	\item Konstante - Ganz- oder Gleitkommazahl (z.B. 9.4)
	\item Variable - Name einer Variable (z.B. x)
	\item + - Addition
	\item - - Subtraktion
	\item * - Multiplikation
	\item / - Division
	\item (;) - Klammer
	\item = - wertet eine Berechung aus
}

\paragraph{Selection Mode:}

Im Selection Mode können mehrere Sprites ausgewählt werden, um mit diesen anschließend eine Operation durchzuführen. Entweder durch einen Klick auf ein Sprite oder durch Ziehen eines Rechtecks um mehrere Sprites können diese ausgewählt werden. Selektierte Sprites werden mit einer blau-strichlierten Linie umrandet.

\paragraph{Move Mode:}

Im Move-Mode können alle selektierten Sprites auf dem Bildschirm verschoben werden. Sie können entweder zuvor mithilfe des Selection-Mode oder im Move-Mode durch einfachen Klick auf ein Sprite ausgewählt werden. Streichen sie von der Ursprungsposition zur Zielposition um die Sprites entsprechend zu verschieben.

\subsubsection{Änderung der Farbe}

Zum Ändern der Strich- und Textfarbe gibt es in der Menubar ein entsprechendes Item. Drücken Sie auf "Set Color" und wählen sie in dem neu geöffneten Fenster durch Streichen über dem Touchscreen eine neue Farbe aus, diese kann durch einen  Klick auf "OK" bestätigt werden. Die aktuell ausgewählte Farbe wird in der Menubar neben dem "Set Color"-Menuitem angezeigt.

\subsubsection{Änderung der Strichstärke}

Zum Ändern der Strichstärke drucken sie auf den Stroke-Dropdown in der Menubar (Strich über das gesamte Item mit einem Pfeil nach unten). Aus der erscheinenden Liste können sie zwischen fünf verschiedenen Strichstärken wählen. Die Strichstärke wird im Draw-Mode berücksichtigt.

\subsubsection{Löschen von Sprites}

Um ein oder mehrere Sprites zu löschen müssen diese vorher ausgewählt werden. Wählen Sie mithilfe des Selection-Mode die zu löschenden Sprites aus und drücken anschließend auf das "Delete"-Menuitem in der Menubar.

\subsubsection{Rückgängigmachen einer Action}

Actions wie das Ändern der Farbe, der Strichstärke, das Löschen von Sprites, etc. kann rückgängig gemacht werden. Durch einen Klick auf "Back" in der Menubar wird der Zustand auf jenen vor der letzten Action zurückgestellt. Falls man in der Historie zu weit zurück geht kann durch Klicken auf "Forward" eine Action wiederhergestellt werden.

\subsubsection{Speichern einer Datei}

Um den aktuellen Zustand des Spritepanels zu Speichern und ihn später wieder herzustellen kann durch einen Klick auf "Save" bzw. "Save as" in der Menubar eine Datei mit den aktuellen Inhalten des Spritepanels erstellt werden.\\
\\
Bei erstmaliger Speicherung und einem Klick auf "Save" bzw. durch einen Klick auf "Save as" muss der Benutzer einen Pfad angeben unter dem die aktuelle Datei gespeichert wird. Bei wiederholter Speicherung in die selbe Datei wird dieser Dialog automatisch übersprungen und der letzte Pfad verwendet.

\subsubsection{Laden einer Datei}

Um den Zustand einer bereits gespeicherten Datei wieder herzustellen kann durch einen Klick auf "Load" eine Datei ausgewählt werden. Nach Auswählen dieser Datei wird der ehemalige Zustand auf dem Spritepanel wieder hergestellt.
\section{Theoretische Grundlagen}
\setcounter{secnumdepth}{4}

\subsection{3D-Druck}
3D-Druck ist nicht gleich 3D-Druck: Der Sammelbegriff steht heute für ein ganzes Bündel von Fertigungstechniken, die nach unterschiedlichen Prinzipien funktionieren und sich jeweils nur für ganz bestimmte Materialien eignen. Ihr gemeinsamer Nenner: Alle Verfahren bauen dreidimensionale Objekte, indem sie Material in dünnen Schichten auftragen und verfestigen. Der Fachbegriff dafür ist additive Fertigung - in Abgrenzung zu subtraktiven Techniken wie Fräsen, Sägen, Bohren oder Wasserstrahlschneiden. \footcite{shit_3d_spiegel}

\subsubsection{Düsen-Technologie}
Mittels sogenanntem Fused Deposition Modeling (FDM) lassen sich nur Materialien verarbeiten, die beim Erhitzen weich und formbar werden - thermoplastische Kunststoffe wie ABS oder PLA, aber auch Modellierwachs und Schokolade. Der Druckkopf von FDM-Maschinen besteht im Kern aus einer heißen Düse, in die das feste Rohmaterial gepresst wird und sich dadurch verflüssigt. Am anderen Ende der Düse tritt es als dünner und weicher Faden aus. Damit zeichnet der Druckkopf eine Schicht des gewünschten Objekts - die äußere Kontur als einfassende Linie, Flächen werden als Schraffuren angelegt. Ist die Schicht vollendet und das Material in der gewünschten Form erstarrt, rückt der Kopf um eine Schichtdicke vom Objekt ab und zeichnet die nächste Lage.\\
\\
Die meisten 3D-Drucker, die weniger als 4000 Euro kosten, arbeiten heute mit FDM, denn die nötige Elektronik und Mechanik für eine ernstzunehmende Maschine dieses Typs ist erstaunlich simpel. Als Rohmaterial dient in der Regel Plastikdraht (Filament) von entweder 1,75 oder 3,0 Millimetern Stärke. Da dieser Draht wie bei einer Heißklebepistole im festen Zustand in die Düse gedrückt wird, sprechen manche Hersteller auch von Fused Filament Fabrication (FFF) - wahrscheinlich einfach deshalb, weil das eine griffigere Abkürzung ergibt.\\
\\
Weil der weiche Plastikdraht an der Luft nicht sofort erstarrt, müssen größere Überhänge und flache Vorsprünge am Objekt während des Drucks abgestützt werden. Bei einfachen FDM-Maschinen fügt die Software an den passenden Stellen der 3D-Vorlage geeignete Gitterstrukturen hinzu, die aus demselben Material wie das eigentliche Objekt aufgebaut werden und die man hinterher von Hand abbrechen, abknipsen oder wegschleifen muss. Gehobene FDM-Maschinen bauen die Stützstrukturen mit einem zweiten Druckkopf und aus einem anderen Material auf. Manche Stützmaterialien sind wasserlöslich oder lassen sich in einem basischen Bad auswaschen.\\
\\
\displayimageg{img/sachsenschnitzel/shit_3d_1}{Hier wurde mit Support-Material gedruckt, das anschließend entfernt werden muss\footcite{shit_3d_1}}
\noindent
Per FDM gefertigte Kunststoffteile sind belastbar und im Vergleich zu anderen additiven Techniken flott gefertigt. Allerdings weist ihre Oberfläche - trotz der heute üblichen Schichtdicken bis hinunter zu 0,1 Millimeter - oft eine deutlich sichtbare Riffelung auf. Bei manchen Objekten erinnert das an eine Holzmaserung und sieht ganz gut aus, bei vielen stört es nicht, gelegentlich wirkt es aber etwas billig.\footcite{shit_3d_spiegel}

\subsubsection{weitere Technologien}
Komplett ohne Stützmaterial kommen alle additiven Techniken aus, die ihr Rohmaterial als Pulver verarbeiten. Ein Hauch davon bildet den Stoff für jede einzelne Schicht des Modells, anschließend geht der Druckkopf darüber und verfestigt das Pulver entsprechend der Form des gewünschten Objekts. Werden Kunststoffe wie Polyamid oder Metalle wie Stahl und Titan verarbeitet, verschmilzt oder sintert ein Laser die einzelnen Körnchen punktgenau, was dann Selective Laser Melting (SLM) oder Selective Laser Sintering (SLS) heißt. Ist stattdessen ein farbiges Modell gewünscht, spritzt ein modifizierter Tintenstrahl-Druckkopf ein Gemisch aus Farbe und Bindemittel in die Pulverschicht und verklebt einzelne Körnchen. Deshalb ist speziell für diese Spielart des Pulverdrucks auch die Bezeichnung 3D-Druck (3DP) gängig, was die Begriffsverwirrung noch erhöht.\\
\\
Alles Pulver, das nicht Teil des Werkstücks wird, bleibt während des Produktionsprozesses liegen. Es stützt überhängende Teile, wird zum Schluss weggebürstet und kann für das nächste Modell wiederverwendet werden. Pulverdrucke haben oft eine raue Oberfläche, ähnlich wie feines Sandpapier. SLS und SLM produzieren robuste, elastische oder filigrane Objekte. Farbdrucker hingegen zaubern realistische Miniaturen, die allerdings etwas spröde wirken. Oft bestehen sie aus einem speziellen Polymergips und sind dadurch recht schwer.\\
\\
Die schönsten Oberflächen erzeugen Stereolithografie-Maschinen (SLA): Das Werkstück nimmt in einem Becken voller flüssigem Kunstharz Gestalt an, dessen Füllstand für jede weitere Schicht minimal erhöht wird. Die Flüssigkeit härtet unter UV-Licht punktuell aus. Entweder zeichnet ein Laser im Druckkopf die nötigen Formen in die Kunstharzoberfläche oder ein Beamer projiziert die komplette Schicht des Modells auf einmal. Stereolithografien zeigen feinste Details, sind oft aber deutlich zerbrechlicher als aus Pulver gesinterte oder gelaserte Objekte. Manche Kunstharze altern zudem sichtbar und verfärben sich dabei.\\
\\
Die Stereolithografie ist die älteste 3D-Druck-Technik und wurde bereits in den Achtzigerjahren entwickelt. Durch die 3D-Blogs geistert seit einiger Zeit der fossile Mitschnitt einer Good-Morning-America-Sendung von 1989, in der eine Stereolithografie-Maschine bei der Arbeit zu sehen ist. Das ist der wohl früheste Fernsehauftritt eines 3D-Druckers und heute noch sehenswert - nicht nur wegen der aus Haarspray in 3D betonierten Frisur der Ansagerin.\footcite{shit_3d_spiegel}

\subsubsection{Spezialtechniken}

Neben Stereolithografie, Pulverdruck und der Fertigung mit heißer Düse gibt es noch einige interessante Spezialmethoden. Manche Drucker der Objet-Serie des Herstellers Stratasys arbeiten nach dem PolyJet-Verfahren, bei dem Kunstharz tröpfchenweise gedruckt und anschließend sofort per UV-Licht gehärtet wird. Der Clou: Mehrere parallel angebrachte Düsen und Köpfe können Objekte aus verschiedenen Materialien in einem Rutsch aufbauen, beispielsweise die harte Schale einer Fernbedienung mit elastischen Tasten. Manche Drucker mischen auch Kunststoffe mit beliebig wählbaren Eigenschaften in puncto Elastizität und Farbe zusammen, was insbesondere für Prototypen bei der Produktentwicklung nützlich ist.\\
\\
Beim 3D-Druck von Edelmetall wie Gold und Silber kommt manchmal ein indirektes Verfahren zum Einsatz: Das Modell wird zunächst aus Modellierwachs in 3D gedruckt und dann konventionell im Wachsausschmelzverfahren mit verlorener Form abgegossen.\\
\\
Die irische Firma Mcor hat für beliebig farbig texturierte Drucke eine Alternative zum spröden Pulver in petto: Ihre Maschinen schichten Modelle aus ganzen Packungen gewöhnlichen Schreibpapiers auf. Für jede Schicht druckt zunächst ein konventioneller Tintenstrahlkopf einen farbigen Horizontalschnitt durch das Objekt aufs Blatt, wobei die Spezialfarbe die gesamte Dicke des Papiers durchdringt. Dann trägt die Maschine auf die Schicht darunter flüssigen Leim in Form des Objekts auf, setzt dann erst das farbig bedruckte Blatt drauf, presst es an und schneidet mit einem Messer den unverklebten Teil des Blatts entlang der Modellkontur ab. Ist das Objekt komplett aufgebaut, wird das überschüssige Papier entfernt und das Werkstück noch mal in Kunstharz getränkt. Das verleiht dem Druck eine haptisch angenehme, seidenglänzende Oberfläche, die nicht mehr im Entferntesten an Papier erinnert. In den Niederlanden kann man solche Drucke seit Kurzem bei der Büromaterialhandelskette Staples fertigen lassen. \footcite{shit_3d_spiegel}
\subsection{Arduino}

Arduino ist eine aus Soft- und Hardware bestehende Physical-Computing-Plattform. Beide Komponenten sind open-source und somit quelloffen. Die Hardware besteht aus einem einfachen Input/Output-Board mit einem Mikrocontroller und analogen und digitalen Ein- und Ausgängen. Die Entwicklungsumgebung soll auch technisch weniger Versierten den Zugang zur Programmierung und zu Mikrocontrollern erleichtern. Die Programmierung selbst erfolgt in C bzw. C++, wobei technische Details wie Header-Dateien vor den Anwendern weitgehend verborgen werden und umfangreiche Bibliotheken und Beispiele die Programmierung vereinfachen. Arduino kann verwendet werden, um eigenständige interaktive Objekte zu steuern oder um mit Softwareanwendungen auf Computern zu interagieren.

\subsubsection{Hardware}
Die Hardware eines typischen Arduino basiert auf einem Atmel-AVR-Mikrocontroller aus der megaAVR-Serie. Abweichungen davon gibt es unter anderem bei den Arduinoboards Arduino Due (ARM Cortex-M3 32-Bit-Prozessor vom Typ Atmel SAM3X8E), Yún, Tre, Gemma und Zero, wo andere Mikrocontroller von Atmel eingesetzt werden. Eine Besonderheit stellen zudem die Arduinoboards Yún und Tre dar, die zusätzlich zum Mikrocontroller einen stärkeren Mikroprozessor besitzen. Alle Boards werden entweder über USB (5 V) oder eine externe Spannungsquelle (7–12 V) versorgt und verfügen über eine Taktung von 16-MHz. Es gibt auch Varianten mit 3,3 V-Versorgungsspannung und welche mit abweichender Taktung.\\
\\
Konzeptionell werden alle Boards über eine serielle Schnittstelle programmiert, wenn Reset aktiviert ist. Der Mikrocontroller ist mit einem Bootloader vorprogrammiert, wodurch die Programmierung direkt über die serielle Schnittstelle ohne externes Programmiergerät erfolgen kann. Bei älteren Boards wurde dafür die RS-232-Schnittstelle genutzt. Bei aktuellen Boards geschieht die Umsetzung von USB nach seriell über einen eigens entwickelten USB-Seriell-Konverter.\\
\\
Alle Arduinoboards, bis auf den Arduino Esplora, stellen digitale Input- und Output-Pins (kurz: I/O-Pins) des Mikrocontrollers zur Nutzung für elektronische Schaltungen zur Verfügung. Üblich ist auch, dass eine bestimmte Anzahl dieser Pins PWM-Signale ausgeben können. Zusätzlich stehen dem Benutzer eine bestimmte Anzahl an analogen Eingängen zur Verfügung. Für die Erweiterung werden vorbestückte oder teilweise unbestückte Platinen – sogenannte „Shields“ – angeboten, die auf das Arduino-Board aufsteckbar sind. Es können aber auch z.B. Steckplatinen für den Aufbau von Schaltungen verwendet werden.

\subsubsection{Software}
Arduino bringt eine eigene integrierte Entwicklungsumgebung (IDE). Dabei handelt es sich um eine Java-Anwendung, die für die gängigen Plattformen Windows, Linux und MacOS kostenlos verfügbar ist. Die Arduino-IDE bringt einen Code-Editor mit und bindet gcc als Compiler ein. Zusätzlich werden die avr-gcc-Library und weitere Arduino-Libraries eingebunden, die die Programmierung in C und C++ stark vereinfachen.\\
\\
Für ein funktionstüchtiges Programm genügt es, zwei Funktionen zu definieren:
\begin{itemize}
\item setup() – wird beim Start des Programms (entweder nach dem Übertragen auf das Board oder nach Drücken des Reset-Tasters) einmalig aufgerufen, um z.B. Pins als Eingang oder Ausgang zu definieren.
\item loop() – wird durchgehend immer wieder durchlaufen, solange das Arduino-Board eingeschaltet ist.
\end{itemize}
Hier ein Beispiel für ein Programm, das eine an das Arduino-Board angeschlossene LED blinken lässt:\\
\ \\
\displaycode{
const int ledPin = 13;         // Die LED ist an Pin 13 angeschlossen\\
void setup()\{\\
\blank{1cm}	pinMode(ledPin, OUTPUT);     // legt den LED-Pin als Ausgang fest\\
\}\\
\\
void loop()\{\\
\blank{1cm}	digitalWrite(ledPin, HIGH);  // LED anschalten\\
\blank{1cm}	delay(1000);                 // 1000 Millisekunden warten\\
\blank{1cm}	digitalWrite(ledPin, LOW);   // LED ausschalten\\
\blank{1cm}	delay(1000);                 // weitere 1000 Millisekunden warten\\
\}
}
\subsection{Raspberry Pi 3}

Der Raspberry Pi 3 Model B ist ein handlicher Single-Board-Computer, der auf dem Betriebssystem Raspbian läuft. Raspbian ist eine Linux-Distribution, die auf Debian basiert. Es folgt eine kleine Zusammenfassung, was die Idee des Paspberry Pi ist und wie sie umgesetzt wurde.

\subsubsection{Idee}
Das Motiv hinter der Entwicklung eines preisgünstigen Rechners war die sinkende Anzahl an Informatikstudenten an der Universität Cambridge sowie die jedes Jahr geringeren Programmierkenntnisse der Studienanfänger. Für einen der Gründe hielt man, dass Computer heute in der Regel teuer und komplex sind und Eltern ihren Kindern deswegen häufig verbieten, mit dem Familien-PC zu experimentieren. Man wollte daher Jugendlichen einen günstigen Computer zum Experimentieren und Erlernen des Programmierens an die Hand geben. Dabei hoffte man, dass sie wie in der Anfangszeit der Heimcomputer (z. B. IMSAI 8080, Apple I, Sinclair ZX80) die Computergrundlagen und -programmierung spielerisch erlernen würden.\footcite{shit_raspi}

\subsubsection{Bauteile}
\setcounter{secnumdepth}{4}
\paragraph{Prozessor}\ \\
Der Prozessor der ersten Generation nutzt den ARMv6-Instruktionssatz. Des Weiteren werden die ARM-Instruktionssatz-Erweiterungen Thumb und Java-Bytecode unterstützt (Jazelle). Der Speicher ist über einen 64 Bit breiten Bus angebunden und wird direkt als Package-on-Package auf den Prozessor gelötet.\\
\\
Da die Raspberry Pi Foundation eine Verringerung der Lebensdauer bei Übertaktung befürchtete, wurde der Prozessor zunächst mit einem „Sticky (engl. wörtlich „klebenden“, das bedeutet: nicht rücksetzbaren) Bit“ ausgestattet, welches unwiderruflich gesetzt wird, sobald der Prozessor übertaktet wird und somit ein Erlöschen der Garantie signalisiert. Nachdem ausführliche Tests zeigten, dass sich ein Übertakten auf bis zu 1 GHz kaum auf die Lebensdauer auswirkt, wurde am 19. September 2012 mit einem Treiber-Update die Möglichkeit geschaffen, sowohl Prozessor als auch GPU und Speicher ohne Garantieverlust zu übertakten. Die Frequenz und Spannung wird dabei im Betrieb nur dann erhöht, wenn die Leistung benötigt wird und die Temperatur des Chips nicht über 85 $^{\circ}$C liegt. Das Sticky-Bit wird nur noch gesetzt, wenn stärker als empfohlen übertaktet wird.\\
\\
Ein starkes Untertakten auf bis zu 50 MHz und Verringern der Spannung ist ebenfalls möglich, was vor allem beim Modell A zu einer deutlich reduzierten Leistungsaufnahme führt.\\
\\
In der zweiten Generation kommt ein SoC mit der Bezeichnung BCM2836 ebenfalls vom Hersteller Broadcom zum Einsatz. Der dort in einer Quadcore-Konfiguration eingesetzte ARM Cortex-A7 mit 900 MHz Taktfrequenz nutzt den ARMv7-Instruktionssatz und kommt auf eine Gesamtrechenleistung von 6.840 DMIPS. Dazu ist der Prozessor um Faktor 3 energieeffizienter als sein Vorgänger.\\
\\
In der dritten Generation wird ein BCM2837 von Broadcom eingesetzt. Der verwendete ARM Cortex-A53 mit 1,2 GHz Taktfrequenz hat 50–60 \% mehr Leistung als die zweite Generation bzw. fast die zehnfache Leistung gegenüber der ersten Generation.\footcite{shit_raspi}

\paragraph{Grafik}\ \\
Der ARM11-Prozessor ist mit Broadcoms „VideoCore“-Grafikkoprozessor kombiniert. OpenGL ES 2.0 wird unterstützt, und Filme in Full-HD-Auflösung (1080p30 H.264 high-profile) können dekodiert und über die HDMI-Buchse und FBAS-Cinchbuchse ausgegeben werden.\\
\\
Am 24. August 2012 wurde bekanntgegeben, dass Lizenzen für das hardwarebeschleunigte Dekodieren von VC1- und MPEG-2-kodierten Videos zusätzlich erworben werden können. Die Lizenz beschränkt sich dabei auf den bei der Bestellung mit der Seriennummer spezifizierten Raspberry Pi, so dass für jeden dieser Mikrorechner eine eigene Lizenz erforderlich ist. Die vorhandene Lizenz zum Dekodieren von H.264-kodierten Videos erlaubt nach Angaben der Raspberry Pi Foundation auch das Kodieren solcher Videos.\\
\\
Im März 2014 legte Broadcom Dokumentation und Treibercode für den SoC BCM21553 unter einer BSD-Lizenz offen, mit dem auch ein freier Grafiktreiber für den verwendeten BCM2835 erstellt werden kann. Ein entsprechender Treiber wurde nach einem von der Raspberry Pi Foundation ausgerufenen und mit 10.000 USD dotierten Programmierwettbewerb im März 2014 von einem einzelnen Programmierer veröffentlicht.\footcite{shit_raspi}

\paragraph{Audio}\ \\

Das Audiosignal erzeugt das System-on-Chip (SoC) BCM 2835 von Broadcom durch eine einfache Pulsweitenmodulation (PWM) und gibt es über den Audioausgang der 3,5-mm-Klinkenbuchse aus. Auf einen echten Digital-Analog-Umsetzer (DAC) wurde aus Kostengründen verzichtet. Diese Lösung gilt jedoch als schwach, weil DAC und Tiefpassfilter fehlen, da ohne diese störende Nebengeräusche, die als Vielfaches der Modulationsfrequenz entstehen, nicht beseitigt werden. Elektrisch ist dieser Ausgang besser zum Anschluss von Aktivboxen oder am Verstärker einer herkömmlichen Stereoanlage geeignet als für verstärkerlose Kopfhörer. Des Weiteren wird ein Audiosignal in digitaler Form über den HDMI-Ausgang ausgegeben.\\
\\
Verschiedene Dritthersteller bieten daher auch dedizierte Audiolösungen in Form von USB-Audio-Karten oder als Aufsteckkarten an, die eine simulierte I$^2$S-Schnittstelle nutzen. Ferner existieren Lösungen, die das Audiosignal aus der HDMI-Schnittstelle extrahieren.\footcite{shit_raspi}

\paragraph{RTC}\ \\

Der Raspberry Pi enthält keine Echtzeituhr. Das Gerät kennt daher nach dem Anschalten weder Datum noch Uhrzeit. Sofern es mit dem Netzwerk verbunden ist und es nicht selbst kritische Teile der Netzwerkinfrastruktur (etwa den Namensdienst) anbietet, kann die Zeit meist via NTP beschafft werden. Ansonsten muss eine separate Echtzeituhr angeschlossen werden, wenn eine verwendete Software die korrekte Uhrzeit benötigt.\footcite{shit_raspi}

\paragraph{GPIO}\ \\

Der Raspberry Pi bietet eine frei programmierbare Schnittstelle (auch bekannt als GPIO, General Purpose Input/Output), worüber LEDs, Sensoren, Displays und andere Geräte angesteuert werden können. Es gibt fünf GPIO-Anschlüsse, wobei im Allgemeinen nur der Anschluss P1 gebraucht wird. Die GPIO-Schnittstelle P1 besteht bei Modell A und Modell B aus 26 Pins und bei Modell A+ und Modell B+ aus 40 Pins, jeweils ausgeführt als doppelreihige Stiftleiste, wovon\\

\begin{itemize}
\item 2 Pins eine Spannung von 5 Volt bereitstellen, aber auch genutzt werden können, um den Raspberry Pi mit Strom zu versorgen,
\item 2 Pins eine Spannung von 3,3 Volt bereitstellen,
\item 1 Pin als Masse dient,
\item 4 Pins, die zukünftig eine andere Belegung bekommen könnten, derzeit ebenfalls mit Masse verbunden sind,
\item 17 Pins (Modell A und B) bzw. 26 Pins (Modell A+ und B+, sowie Raspberry Pi 2 Modell B), welche frei programmierbar sind. \item Sie sind für eine Spannung von 3,3 Volt ausgelegt. Einige von ihnen können Sonderfunktionen übernehmen:
\item 5 Pins können als SPI-Schnittstelle verwendet werden,
\item 2 Pins haben einen 1,8-k$\Omega$-Pull-up-Widerstand (auf 3,3 V) und können als I$^2$C-Schnittstelle verwendet werden,
\item 2 Pins können als UART-Schnittstelle verwendet werden.
\end{itemize}
\noindent

Mit dem Modell B+ wurde eine offizielle Spezifikation für Erweiterungsplatinen, sogenannte Hardware attached on top (HAT), vorgestellt. Jeder HAT muss über einen EEPROM-Chip verfügen; Darin finden sich Herstellerinformationen, die Zuordnung der GPIO-Pins sowie eine Beschreibung der angeschlossenen Hardware in Form eines „device tree“-Abschnitts. Dadurch können die nötigen Treiber für den HAT automatisch geladen werden. Auch die genaue Größe und Geometrie des HAT sowie die Position der Steckverbinder werden dadurch festgelegt. Modell A+ und Raspberry Pi 2 Modell B sind mit diesen ebenfalls kompatibel.\\
\\
Die in der Revision 2 hinzugekommene GPIO-Schnittstelle P6 erlaubt es, den Raspberry Pi zurückzusetzen bzw. zu starten, nachdem er heruntergefahren wurde. Zur Steuerung der GPIOs existieren Bibliotheken für zahlreiche Programmiersprachen. Auch eine Steuerung durch ein Terminal oder Webinterfaces ist möglich.\footcite{shit_raspi}
\subsection{Java}
Java ist eine universelle, objektorientierte und klassenbasierte Programmiersprache. Das Prinzip von Java lautet im original ''write once, run anywhere'', was bedeutet, dass kompilierter Java-Code auf jedem Betriebssystem laufen kann und keine Neukompilierung notwendig ist.\\
\\

\subsubsection{Philosophie}
Der Entwurf der Programmiersprache Java strebte hauptsächlich fünf Ziele an:

\begin{itemize}
	\item Sie soll eine einfache, objektorientierte, verteilte und vertraute Programmiersprache sein.
	\item Sie soll robust und sicher sein.
	\item Sie soll architekturneutral und portabel sein.
	\item Sie soll sehr leistungsfähig sein.
	\item Sie soll interpretierbar, parallelisierbar und dynamisch sein.
\end{itemize}
\noindent


\setcounter{secnumdepth}{4}
\paragraph{Einfachheit}
Java ist im Vergleich zu anderen objektorientierten Programmiersprachen wie C++ oder C\# einfach, da es einen reduzierten Sprachumfang besitzt und beispielsweise Operatorüberladung und Mehrfachvererbung nicht unterstützt.

\paragraph{Objektorientierung}
Java gehört zu den objektorientierten Programmiersprachen.

\paragraph{Verteilt}
Eine Reihe einfacher Möglichkeiten für Netzwerkkommunikation, von TCP/IP-Protokollen über Remote Method Invocation bis zu Webservices werden vor allem über Javas Klassenbibliothek angeboten; die Sprache Java selbst beinhaltet keine direkte Unterstützung für verteilte Ausführung.

\paragraph{Vertrautheit}
Wegen der syntaktischen Nähe zu C++, der ursprünglichen Ähnlichkeit der Klassenbibliothek zu Smalltalk-Klassenbibliotheken und der Verwendung von Entwurfsmustern in der Klassenbibliothek zeigt Java für den erfahrenen Programmierer keine unerwarteten Effekte.

\paragraph{Robustheit}
Viele der Designentscheidungen bei der Definition von Java reduzieren die Wahrscheinlichkeit ungewollter Systemfehler; zu nennen sind die starke Typisierung, Garbage Collection, Ausnahmebehandlung sowie Verzicht auf Zeigerarithmetik.


\paragraph{Sicherheit}
Dafür stehen Konzepte wie der Class-Loader, der die sichere Zuführung von Klasseninformationen zur Java Virtual Machine steuert, und Security-Manager, die sicherstellen, dass nur Zugriff auf Programmobjekte erlaubt wird, für die entsprechende Rechte vorhanden sind.

\paragraph{Architekturneutralität}
Java wurde so entwickelt, dass dieselbe Version eines Programms prinzipiell auf einer beliebigen Computerhardware läuft, unabhängig von ihrem Prozessor oder anderen Hardwarebestandteilen.

\paragraph{Portabilität}
Zusätzlich zur Architekturneutralität ist Java portabel. Das heißt, dass primitive Datentypen sowohl in ihrer Größe und internen Darstellung als auch in ihrem arithmetischen Verhalten standardisiert sind. Beispielsweise ist ein float immer ein IEEE 754 Float von 32 Bit Länge. Dasselbe gilt beispielsweise auch für die Klassenbibliothek, mit deren Hilfe man eine vom Betriebssystem unabhängige GUI erzeugen kann.

\paragraph{Leistungsfähigkeit}
Java hat aufgrund der Optimierungsmöglichkeit zur Laufzeit das Potential, eine bessere Performance als auf Compilezeit-Optimierungen begrenzte Sprachen (C++, etc) zu erreichen. Dem entgegen steht der Overhead durch die Java-Laufzeitumgebung, sodass die Leistungsfähigkeit von beispielsweise C++-Programmen in einigen Kontexten übertroffen, in anderen aber nicht erreicht wird.

\paragraph{Interpretierbarkeit}
Java wird in maschinenunabhängigen Bytecode kompiliert, dieser wiederum kann auf der Zielplattform interpretiert werden. Die Java Virtual Machine der Firma Oracle (früher Sun) interpretiert Java-Bytecode, bevor sie ihn aus Performancegründen kompiliert und optimiert.

\paragraph{Parallelisierbarkeit}
Java unterstützt Multithreading, also den parallelen Ablauf von eigenständigen Programmabschnitten. Dazu bietet die Sprache selbst die Schlüsselwörter synchronized und volatile – Konstrukte, die das „Monitor \& Condition Variable Paradigma“ von C. A. R. Hoare unterstützen. Die Klassenbibliothek enthält weitere Unterstützungen für parallele Programmierung mit Threads. Moderne JVMs bilden einen Java-Thread auf Betriebssystem-Threads ab und profitieren somit von Prozessoren mit mehreren Rechenkernen.

\paragraph{Dynamik}
Java ist so aufgebaut, dass es sich an dynamisch ändernde Rahmenbedingungen anpassen lässt. Da die Module erst zur Laufzeit gelinkt werden, können beispielsweise Teile der Software (etwa Bibliotheken) neu ausgeliefert werden, ohne die restlichen Programmteile anpassen zu müssen. Interfaces können als Basis für die Kommunikation zwischen zwei Modulen eingesetzt werden; die eigentliche Implementierung kann aber dynamisch und beispielsweise auch während der Laufzeit geändert werden. 

\subsubsection{Konzepte}

\paragraph{Objekte}
Um das Prinzip der Objektorientierung verstehen zu können, muss man sich zunächst vor Augen halten, was ein Objekt genau ist:\\
\\
Nehmen wir uns zur Veranschaulichung ein Beispiel:\\
Unser Beispielobjekt ist ein Kugelschreiber, den jeder zu Hause hat. Dabei ist bereits erkennbar, dass ein Objekt ein Gegenstand im realen Leben sein kann, im direkten Umfeld. Dieses Kugelschreiberobjekt hat gewisse Eigenschaften. Im Fachjargon werden die Eigenschaften auch als Attribute bezeichnet. Die Attribute beschreiben das Objekt. Wenn man den Kugelschreiber betrachtet, ist ein Attribut direkt ersichtlich, nämlich die Farbe. Das Attribut ''Farbe'' in unserem Beispiel soll beispielsweise weiß sein. Eine weitere Eigenschaft ist die Schreibfarbe, die bei meinem Kugelschreiber schwarz ist. Jetzt haben wir schon zwei Attribute gefunden, die unser Kugelschreiberobjekt beschreiben.\\
\\
Man kann aber mit diesem Kugelschreiberobjekt auch noch interagieren. Die Kugelschreibermine ist ausfahrbar, indem auf den Druckknopf oben gedrückt wird. Interaktionen, die man an einem Objekt durchführen kann, nennt man beim Programmieren Methoden oder Funktionen (in Java vorzugsweise ''Methode''). Auch diese gehören zu unserem Kugelschreiber-Objekt.\\
\\
In der Softwaretechnik benutzt man Funktionen in der Regel, um auf die Eigenschaften (Attribute) zuzugreifen und  diese zu setzen bzw. zu verändern. Um dies näher zu erläutern, wenden wir uns wieder unserem Kugelschreiber zu: Eine weitere Eigenschaft unseres Kugelschreibers ist der Status des Zustandes der Kugelschreibermine. Diese bezeichnen wir hier mit dem Namen ''ausgefahren''. Diese Eigenschaft kann zwei Zustände haben nämlich WAHR (im Programmierjargon true) für ''JA, sie ist ausgefahren'' und FALSCH (false) für ''NEIN, der Kugelschreiber ist nicht ausgefahren''. Eine Zugriffsmethode könnten wir z.B. ''istAusgefahren'' nennen. Mit dieser Methode fragen wir den Status der Kugelschreibermine ab. Damit wir den Status der Eigenschaft ''ausgefahren'' verändern können, benötigen wir eine weitere Methode, die wir ''fahreKugelschreibermineEinAus'' nennen. Mit dieser Methode kann ich z.B. den Status der Eigenschaft ''ausgefahren'' verändern. Diese Veränderung ist abhängig von dem vorigen Status der Kugelschreibermine, deswegen überprüfen wir anhand der Methode ''istAusgefahren'' den Status der Kugelschreibermine und ändern den Status von true auf false bzw. umgekehrt.\\
\\
Nicht immer sind Objekte jedoch greifbare Gegenstände aus dem realen Leben. Oft bildet man auf diese Weise Datenstrukturen ab. Ein Beispiel dafür ist die Adresse. Auch die Adresse wird in der objektorientierten Programmierung zu einem Objekt. Die Eigenschaften sind in dem Fall Straße, Hausnummer, Postleitzahl und Ort. Eine Funktion könnte z.B. darin bestehen, die Eigenschaften nach einem Umzug zu ändern.\\
\\
Die Attribute können zusätzlich selber Objekte sein. Beispielsweise könnte bei der Realisierung einer Kundenverwaltung der Kunde neben seinen Attributen Kundennummer, Name usw. auch die Adresse als Attribut haben, die, wie eben dargestellt, selber als Objekt aufgebaut sein kann.

\paragraph{Klassen}
Ein Objekt ist eine genaue Beschreibung z.B. unseres Kugelschreibers. Die Attribute (Eigenschaften) haben  einen Wert bzw. einen festen Zustand. Unser Kugelschreiber hatte beispielsweise die Außenfarbe weiß und die Schreibfarbe schwarz.\\
\\
Die Klasse hingegen ist eine Verallgemeinerung aller Kugelschreiberobjekte. Sie beschreibt, welche Attribute ein Kugelschreiber haben kann ohne diesen einen Wert zuzuweisen. Erst das Objekt repräsentiert den Gegenstand Kugelschreiber, den wir vor uns liegen haben mit  all seinen Eigenschaften.\\
\\
Ein weiteres Beispiel ist die Klasse ''Mensch''. Bereits vor der Geburt weiß man, über welche Eigenschaften (z.B. Augenfarbe, Größe, Geschlecht) und Funktionen (z.B. Sprechen) ein Mensch später verfügen wird, man weiß jedoch nicht, welchen Wert diese haben. Eine einzelne Person entspricht dann einem Objekt der Klasse ''Mensch'' (auch wenn der Begriff Objekt in dem Zusammenhang vielleicht etwas unpassend ist), dessen Eigenschaften bei der Entstehung gesetzt werden (z.B. Augenfarbe=braun).\\
\\
In der objektorientierten Programmierung  gehört also jedes Objekt zu einer Klasse. Dieses besitzt die Attribute (Eigenschaften) und die Methoden (Interaktionen) dieser Klasse.\\
\\
In der Softwaretechnik spricht man auch davon, dass Klassen einen eigenen Datentypen (siehe Glossar o. Kapitel Java-Datentypen) bilden. Da jedes Attribut durch einen Datentyp beschrieben wird, kann eine Klasse auch Attribute beinhalten, die selber wieder Objekte einer anderen oder sogar der eigenen Klasse sind. Als Beispiel hatten wir im Kapitel ''Objekt'' die Klasse Adresse angeführt, die die Attribute Straße, Hausnummer, PLZ und Ort hat. Unsere Klasse Mensch kann nun zur besseren Strukturierung der Daten  ein Attribut ''Anschrift'' beinhalten, das wiederum ein Objekt der Klasse Adresse ist.\\
\\
Wenn wir bei der Klasse Mensch bleiben, könnte man sich z.B. zum Aufbau eines Stammbaums außerdem die Attribute ''Vater'' und ''Mutter'' denken, die selber auch wieder zur Klasse Mensch gehören und ebenfalls mit Attributen wie Name und Adresse ausgestattet sind.
 
 \paragraph{Vererbung}
Der Begriff Vererbung hört sich erst einmal sehr familiär an. Man kann im weitesten Sinne unter Vererbung verstehen, dass etwas weiter gereicht wird. Im familiären Umfeld bekommen also Kinder etwas von Ihren Eltern oder gar Großeltern vererbt.\\
\\
In der Objektorientierung wird eine Kindsklasse als Child- oder Subklasse bezeichnet, wohingegen die Elternklasse Parent- oder Superklasse genannt wird. Man spricht in der Objektorientierung von Vererbung, wenn eine Subklasse Attribute und Methoden von der Superklasse vererbt bekommt. Eine Subklasse besitzt sowohl alle Attribute und Methoden der Superklasse als auch eigene speziellere Attribute und Methoden, die die Subklasse näher beschreiben.\\
\\
Die Superklasse ist eben sehr allgemein gehalten, wohingegen die Subklasse eine spezielle Ausprägung der Superklasse ist. Eine Superklasse kann beliebig viele Subklassen haben. Als Beispiel lässt sich auch hier der Kugelschreiber verwenden. Eine Superklasse könnte hier ''Stift'' lauten, der bereits die Attribute ''Stiftfarbe'' und ''Schreibfarbe'' hat. Die Kugelschreiber-Klasse ist eine Child-Klasse der Klasse ''Stift'' und erbt diese Attribute. Zusätzlich kommen noch Funktionen für das Ein- und Ausfahren der Kugeschreibermine hinzu. Die Klasse ''Kugelschreiber'' könnte selber wiederum eine Superklasse von ''Drehkugelschreiber'' und ''Druckkugelschreiber'' sein. Die Hierarchie kann beliebig tief sein.\\
\\
In der Programmiersprache Java kann allerdings eine Subklasse nur genau eine Superklasse besitzen. In der Programmiersprache C++ ist dies anders, da dort eine Subklasse von mehreren Superklassen erben kann.\\
\\
\paragraph{Interfaces}
Um die Mehrfachvererbung in Java zu umgehen, wurden als eine Art Kompromiss Interfaces eingeführt. Ein Interface ist eine Schnittstelle, in der festgelegt wird, über welche Methoden die Klassen, die das Interface implementieren, verfügen müssen. Die Interfaces selber enthalten daher nur Funktionsköpfe und Konstanten. Alle Klassen, die das Interface implementieren, müssen sämtliche Methoden, die das Interface vorgibt, enthalten.\\
\\
Interfaces werden verwendet, um Gemeinsamkeiten (z.B. gleiche Funktionalitäten), die mehreren Klassen zugrunde liegen, in einer separaten Klasse zu definieren. Die Objekte der implementierenden Klasse sind wie bei der Vererbung gleichzeitig auch Objekte des Interfaces. In Java werden Interfaces daher oft genutzt, um die fehlende Mehrfachvererbung gewissermaßen zu simulieren, da eine Klasse zwar nur von einer Superklasse abgeleitet werden kann, jedoch beliebig viele Interfaces implementieren kann.\\
\\
In der Praxis werden Interfaces häufig für Kommunikationszwecke verwendet. Zwei miteinander kommunizierende Seiten besitzen beispielsweise ein festgelegtes Interface, damit eine reibungslose Kommunikation durchgeführt werden kann. So wird gewährleistet, dass beide Seiten die vom Interface vorgegebenen Methoden implementieren.\\
\\
Das Interface dient auch dazu, den eigenen Quellcode vor fremden Entwicklern zu schützen, da diese nur auf die Methoden des Interfaces zugreifen können.

\paragraph{Arrays}
Eindimensionale Arrays (deutsch: Felder) sind im Prinzip einfache Listen. Diese Arrays werden mit einem Datentypen deklariert, d.h. alle Werte, die in diesem Array gespeichert werden sollen, müssen von demselben Datentyp sein, mit dem das Array deklariert wurde.\\
\\
Das Array erhält außerdem eine festgelegte unveränderbare Größe, die in dem Attribut length gespeichert wird und zur Laufzeit abfragbar ist. Wird bei einem Array z.B. auf ein elftes Element zugegriffen und das Array wurde mit einer Länge von zehn deklariert, so wird eine sogenannte \displaycode{java.lang.ArrayIndexOutOfBoundsException} (Ausnahme) geworfen. \\
\\
Kennzeichnend für ein Array sind die eckigen Klammern. Diese können wahlweise zwischen Datentyp und Arrayname oder nach dem Arraynamen stehen. Da es sich bei einem Array um einen komplexen Datentyp handelt, benötigt man bei der Erzeugung des Arrays den \displaycode{new}-Operator.\\
\\
Um ein bestimmtes Element des Arrays ansprechen zu können, hat jedes Element eine Nummer, den sogenannten Index. Das erste Element bekommt den Index 0, das zweite den Index 1 usw. Mit der sogenannten Initialisierungsliste ist es möglich, bereits bei der Erstellung des Arrays den Elementen in einem Schritt sofort Werte zuzuweisen. Hier ist dann kein \displaycode{new}-Operator erforderlich.\\

\displayownimageg{img/sachsenschnitzel/shit_code_1}{Beispiele für die Initialisierung von Arrays}{Florian Weinzerl}
\ \\
In dem obigen Beispiel haben wir mit dem new-Operator ein Array erzeugt, das die Größe 5 hat und dessen Elemente vom Datentyp int sind. Das Array enthält jedoch noch keine Werte. Die zweite Erzeugung erfolgt über die  Initialisierungsliste. Dabei wird direkt jedes Feld-Element eines Arrays mit einem Wert belegt.\\
\\
Um auf ein Element zuzugreifen, muss nach dem Array-Namen in eckigen Klammer der Index des Elementes angegeben werden.\\

\displayownimageg{img/sachsenschnitzel/shit_code_2}{Beispiele für den Elementzugriff in Arrays}{Florian Weinzerl}
\ \\
In der ersten Zeile erzeugen wir durch Initialisierungsliste ein String-Array mit fünf Namen. In der zweiten Zeile speichern wir das Element mit dem Index 3 in der Variablen name ab. In diesem Fall enthält die Variable den Namen "Maria" (Nicht vergessen: Bei 0 wird zu zählen begonnen!).
\subsection{Python}\displayauthor{Michael Friesenhengst}\ \\
\noindent
Python ist eine einfach zu lernende, aber mächtige Programmiersprache mit effizienten abstrakten Datenstrukturen und einem einfachen, aber effektiven Ansatz zur objektorientierten Programmierung. Durch die elegante Syntax und die dynamische Typisierung ist Python als interpretierte Sprache sowohl für Skripte als auch für schnelle Anwendungsentwicklung hervorragend geeignet.\\
\\
Der Python-Interpreter und die umfangreiche Standardbibliothek sind als Quelltext und in binärer Form für alle wichtigen Plattformen auf der Webseite http://www.python.org frei verfügbar, und können frei weiterverbreitet werden. Auf der gleichen Seite finden sich Distributionen von Drittanbietern, Verweise auf weitere freie Module, Programme und Werkzeuge, sowie Dokumentation.\\
\\
Der Python-Interpreter kann auf einfache Weise um neue Funktionen und Datentypen erweitert werden, die in C oder C++ (oder andere Sprachen, die sich von C aus ausführen lassen) implementiert sind. Auch als Erweiterungssprache für anpassbare Applikationen ist Python hervorragend geeignet.\footcite{python_tutorial_intro}


\subsubsection{Warum Python?}

Wer viel am Computer arbeitet, kommt irgendwann zu dem Schluss, dass es Aufgaben gibt, die er gern automatisieren würde. Beispielsweise ein Suchen-und-Ersetzen für eine Vielzahl von Dateien oder eine Möglichkeit, einen Haufen Fotodateien auf komplizierte Art umzubenennen oder umzuräumen. Oder man hätte gerne eine kleine Datenbank nach Maß, eine spezialisierte GUI-Anwendung oder ein einfaches Spiel.\\
\\
Als professioneller Softwareentwickler muss man vielleicht mit mehreren C/C++/Java-Bibliotheken arbeiten, findet aber den üblichen Schreiben/Kompilieren/Testen/Re-Kompilieren-Zyklus zu langsam. Wer eine Testsuite für solch eine Bibliothek schreibt, hält es vielleicht für eine ermüdende Aufgabe, den Testcode zu schreiben. Vielleicht hat der ein oder andere auch ein Programm geschrieben, das eine Erweiterungssprache gebrauchen könnte, will aber keine ganz neue Sprache für sein Programm entwerfen und implementieren. Dann ist Python genau die richtige Sprache!\\
\\
Man könnte natürlich Unix-Shellskripte oder Windows-Batchdateien für ein paar dieser Aufgaben schreiben. Mit Shellskripten lassen sich gut Dateien verschieben und Textdaten verändern, zur Entwicklung von GUI-Applikationen oder Spielen sind sie aber weniger geeignet. Man könnte ein entsprechendes C/C++/Java-Programm dafür schreiben, aber es kostet in der Regel bereits viel Entwicklungszeit, um überhaupt einen ersten Programmentwurf zu entwickeln. Python ist einfacher zu nutzen, verfügbar für Windows-, Mac OS X- und Unix-Betriebssysteme und hilft, die Aufgabe schneller zu erledigen.\\
\\
Python ist einfach in der Anwendung, aber eine echte Programmiersprache, die viel mehr Struktur und Unterstützung für große Programme bietet, als Shellskripte oder Batchdateien es könnten. Auf der anderen Seite bietet Python auch mehr Fehlerüberprüfungen als C und hat, als stark abstrahierende Hochsprache, mehr abstrakte Datentypen wie flexible Arrays und Wörterbücher (Dictionaries) eingebaut.\\
\\
Python erlaubt die Aufteilung von Programmen in Module, die in anderen Python-Programmen wiederverwendet werden können. Es kommt mit einer großen Sammlung von Standardmodulen, die als Grundlage für eigene Programme genutzt werden können; oder als Beispiele, um in Python Programmieren zu lernen. Manche der Module stellen Datei-I/O, Systemaufrufe, Sockets und sogar Schnittstellen zu GUI-Toolkits bereit.\\
\\
Python ist eine interpretierte Sprache, wodurch sich bei der Programmentwicklung erheblich Zeit sparen lässt, da Kompilieren und Linken nicht nötig sind. Der Interpreter kann interaktiv genutzt werden, so dass man einfach mit den Fähigkeiten der Sprache experimentieren oder Wegwerf-Code schreiben kann. Es ist auch ein praktischer Tischrechner.\\
\\
Python ermöglicht die Entwicklung von kompakten und lesbaren Programmen. Programme, die in Python geschrieben sind, sind aus mehreren Gründen viel kürzer als C/C++/Java-Äquivalente:\\

\begin{itemize}
\item Die abstrakten Datentypen erlauben es, komplexe Operationen in einer einzigen Anweisung auszudrücken;
\item Anweisungen werden durch Einrückungen und nicht durch öffnende und schließende Klammern gruppiert
\item Variablen- oder Argumentdeklarationen sind nicht nötig.
\end{itemize}
Python ist erweiterbar: Wer in C programmieren kann, kann einfach eine neue eingebaute Funktion oder ein Modul zum Interpreter hinzuzufügen. Entweder um zeitkritische Operationen mit maximaler Geschwindigkeit auszuführen oder um Python-Programme zu Bibliotheken zu linken, die nur in binärer Form (wie beispielsweise herstellerspezifische Grafikbibliotheken) verfügbar sind. Wenn man erst einmal mit Python vertraut ist, kann man den Python-Interpreter zu in C geschriebenen Applikationen linken und Python als Erweiterung oder Kommandosprache für diese Applikation nutzen.\footcite{python_tutorial_appetite}

\subsubsection{Datentypen und Strukturen}
Python besitzt eine größere Anzahl von grundlegenden Datentypen. Neben der herkömmlichen Arithmetik unterstützt es transparent auch beliebig große Ganzzahlen und komplexe Zahlen.\\
\\
Die Sprache verfügt über die übliche Ausstattung an Zeichenkettenoperationen. Zeichenketten sind in Python allerdings unveränderliche Objekte (wie auch in Java). Daher geben Operationen, die das Ändern einer Zeichenkette bewerkstelligen sollen – wie z. B. das Ersetzen von Zeichen – immer eine neue Zeichenkette zurück.\\
\\
In Python ist alles ein Objekt; Klassen, Typen, Methoden, Module etc. Der Datentyp ist jeweils an das Objekt (den Wert) gebunden und nicht an eine Variable, d. h. Datentypen werden dynamisch vergeben – nicht wie bei Java.\\
\\
Trotz der dynamischen Typverwaltung enthält Python eine gewisse Typprüfung. Implizite Umwandlungen nach dem Duck-Typing-Prinzip sind unter anderem für numerische Typen definiert, so dass man beispielsweise eine komplexe Zahl mit einer langen Ganzzahl ohne explizite Typumwandlung multiplizieren kann. Mit dem Format-Operator \displaycode{\%} gibt es eine implizite Umwandlung eines Objekts in eine Zeichenkette. Der Operator \displaycode{==} überprüft zwei Objekte auf (Wert-)Gleichheit. Der Operator \displaycode{is} überprüft die tatsächliche Identität zweier Objekte.\footcite{python_wiki}

\paragraph{Sammeltypen}\ \\
Python besitzt mehrere Sammeltypen, darunter Listen, Tupel, Mengen (Sets) und assoziative Arrays (Dictionaries). Listen, Tupel und Zeichenketten sind Folgen (Sequenzen, Arrays) und kennen fast alle die gleichen Methoden: Über die Zeichen einer Kette kann man ebenso iterieren wie über die Elemente einer Liste. Außerdem gibt es die unveränderlichen Objekte, die nach ihrer Erzeugung nicht mehr geändert werden können. Listen sind z. B. erweiterbare Felder (Arrays), wohingegen Tupel und Zeichenketten eine feste Länge haben und unveränderlich sind.\\
\\
Der Zweck solcher Unveränderlichkeit hängt z. B. mit den Wörterbüchern zusammen, einem Datentyp, der auch als assoziatives Array bezeichnet wird. Um die Datenkonsistenz zu sichern, müssen die Schlüssel eines Wörterbuches vom Typ „unveränderlich“ sein. Die ins Wörterbuch eingetragenen Werte können dagegen von beliebigem Typ sein.\\
\\
Sets sind Mengen von Objekten und in CPython ab Version 2.4 im Standardsprachumfang enthalten. Diese Datenstruktur kann beliebige (paarweise unterschiedliche) Objekte aufnehmen und stellt Mengenoperationen wie beispielsweise Durchschnitt, Differenz und Vereinigung zur Verfügung.\footcite{python_wiki}

\paragraph{Objektsystem}\ \\

Das Typsystem von Python ist auf das Klassensystem abgestimmt. Obwohl die eingebauten Datentypen genau genommen keine Klassen sind, können Klassen von einem Typ erben. So kann man die Eigenschaften von Zeichenketten oder Wörterbüchern erweitern – auch von Ganzzahlen. Python unterstützt Mehrfachvererbung.\\
\\
Die Sprache unterstützt direkt den Umgang mit Typen und Klassen. Typen können ausgelesen (ermittelt) und verglichen werden und verhalten sich wie Objekte – in Wirklichkeit sind die Typen selbst ein Objekt. Die Attribute eines Objektes können als Wörterbuch extrahiert werden.\footcite{python_wiki}

\subsubsection{Syntax}
Eines der Entwurfsziele für Python war die gute Lesbarkeit des Quellcodes. Die Anweisungen benutzen häufig englische Schlüsselwörter, wo andere Sprachen Symbole einsetzen. Darüber hinaus besitzt Python weniger syntaktische Konstruktionen als viele andere strukturierte Sprachen wie C, Perl oder Pascal:\\
\begin{itemize}
\item zwei Schleifenformen
	\begin{itemize}
		\item for zur Iteration über die Elemente einer Sequenz
		\item while zur Wiederholung einer Schleife, solange ein logischer Ausdruck wahr ist.
	\end{itemize}
\item Verzweigungen
	\begin{itemize}
		\item if … elif … else für Verzweigungen
	\end{itemize}
\end{itemize}
\ \\
Beim letzten Punkt bieten andere Programmiersprachen zusätzlich \displaycode{switch} und/oder \displaycode{goto}. Diese wurden zugunsten der Lesbarkeit in Python weggelassen und müssen durch \displaycode{if}-Konstrukte oder andere Verzweigungsmöglichkeiten (Slices, Wörterbücher) abgebildet werden. Im Gegensatz zu vielen anderen Sprachen können \displaycode{for}- und \displaycode{while}-Schleifen einen \displaycode{else}-Zweig haben. Dieser wird nur ausgeführt, wenn die Schleife vollständig durchlaufen wurde und nicht mittels \displaycode{break} abgebrochen wird.\footcite{python_wiki}

\paragraph{Strukturierung durch Einrücken}\ \\
Python benutzt Einrückungen als Strukturierungselement. Diese Idee wurde erstmals von Peter J. Landin vorgeschlagen und von ihm off-side rule („Abseitsregel“) genannt. In den meisten anderen Programmiersprachen werden Blöcke durch Klammern oder Schlüsselwörter markiert, während verschieden große Leerräume außerhalb von Zeichenketten keine spezielle Semantik tragen. Bei diesen Sprachen ist die Einrückung zur optischen Hervorhebung eines Blockes zwar erlaubt und in der Regel auch erwünscht, aber nicht vorgeschrieben. Für Programmierneulinge wird der Zwang zu lesbarem Stil aber als Vorteil gesehen.\\
\\
Hierzu als Beispiel die Berechnung der Fakultät einer Ganzzahl, einmal in C und einmal in Python:\\
\\
Fakultätsfunktion in C:\\
\\
\displaycode{
int fakult(int x)\{\\
\blank{1cm} if (x > 1)\\
\blank{2cm} return x * fakult(x - 1);\\
\blank{1cm} else\\
\blank{2cm} return 1;\\
\}
}\\
\\
Die gleiche Funktion in Python:\\
\\
\displaycode{
def fakult(x):\\
\blank{1cm}if x > 1:\\
\blank{1cm}\blank{1cm}return x * fakult(x - 1)\\
\blank{1cm}else:\\
\blank{1cm}\blank{1cm}return 1\\
}\\
\\
Es ist jedoch darauf zu achten, die Einrückungen im gesamten Programmtext gleich zu gestalten. Die gemischte Verwendung von Leerzeichen und Tabulatorzeichen kann zu Problemen führen, da der Python-Interpreter Tabstops im Abstand von acht Leerzeichen annimmt. Je nach Konfiguration des Editors können Tabulatoren optisch mit weniger als acht Leerzeichen dargestellt werden, was zu Syntaxfehlern oder ungewollter Programmstrukturierung führen kann.\footcite{python_wiki}\\
\\
Man kann die Fakultätsfunktion aber auch wie in C einzeilig mit ternärem Operator formulieren:\\
\\
Die Fakultätsfunktion in C:\\
\\
\displaycode{
int fakult(int x)\{\\
\blank{1cm}return (x > 1) ? (x * fakult(x - 1)) : 1;\\
\}\\
}\\
\\
Die Fakultätsfunktion in Python:\\
\\
\displaycode{
def fakult(x):\\
\blank{1cm}return x * fakult(x - 1) if x > 1 else 1\\
}\\

\paragraph{Funktionales Programmieren}\ \\

Ausdrucksstarke syntaktische Elemente zur funktionalen Programmierung vereinfachen das Arbeiten mit Listen und anderen Sammeltypen. Eine solche Vereinfachung ist die Listennotation, die aus der funktionalen Programmiersprache Haskell stammt.\\
\\
Hier bei der Berechnung der ersten fünf Zweierpotenzen:
\ \\
\displaycode{
zahlen = [1, 2, 3, 4, 5]\\
zweierpotenzen = [2 ** n for n in zahlen]
}\\
\\
Weil in Python Funktionen als Argumente auftreten dürfen, kann man auch ausgeklügeltere Konstruktionen ausdrücken, wie den Continuation-passing style.\\
\\
Pythons Schlüsselwort lambda könnte manche Anhänger der funktionalen Programmierung fehlleiten. Solche lambda-Blöcke in Python können nur Ausdrücke enthalten, aber keine Anweisungen. Damit werden solche Anweisungen generell nicht verwendet, um eine Funktion zurückzugeben. Die übliche Vorgehensweise ist stattdessen, den Namen einer lokalen Funktion zurückzugeben. Das folgende Beispiel zeigt dies anhand einer einfachen Funktion nach den Ideen von Haskell Brooks Curry:\\
\\
\displaycode{
def add\_and\_print\_maker(x):\\
\blank{1cm}def temp(y):\\
\blank{1cm}\blank{1cm}print("\{\} + \{\} = \{\}".format(x, y, x + y))\\
\blank{1cm}return temp\\
}\\
\\
Damit ist auch Currying auf einfache Art möglich, um generische Funktionsobjekte auf problemspezifische herunterzubrechen. Hier ein einfaches Beispiel:\\
\\
\displaycode{
def curry(func, knownargument):\\
\blank{1cm}return lambda unknownargument: func(unknownargument, knownargument)\\
}\\
Wird die curry-Funktion aufgerufen, erwartet diese eine Funktion mit zwei notwendigen Parametern sowie die Parameterbelegung für den zweiten Parameter dieser Funktion. Der Rückgabewert von curry ist eine Funktion, die dasselbe tut wie func, aber nur noch einen Parameter benötigt.\\
\\
Anonyme Namensräume (sog. Closures) sind mit den o. g. Mechanismen in Python ebenfalls einfach möglich. Ein simples Beispiel für einen Stack, intern durch eine Liste repräsentiert:\\
\\
\displaycode{
def stack():\\
\blank{1cm}l = []\\
\\
\blank{1cm}def pop():\\
\blank{1cm}\blank{1cm}if not is\_empty():\\
\blank{1cm}\blank{1cm}\blank{1cm}return l.pop()\\
\\
\blank{1cm}def push(element):\\
\blank{1cm}\blank{1cm}l.append(element)\\
\\
\blank{1cm}def is\_empty():\\
\blank{1cm}\blank{1cm}return len(l) == 0\\
\\
\blank{1cm}return pop, push, is\_empty\\
\\
pop, push, is\_empty = stack()\\
}\\
\\
Auf diese Weise erhält man die drei Funktionsobjekte pop, push, is\_empty, um den Stack zu modifizieren bzw. auf enthaltene Elemente zu prüfen, ohne l direkt modifizieren zu können.\footcite{python_wiki}\\

\paragraph{Ausnahmebehandlung}\ \\

Python nutzt ausgiebig die Ausnahmebehandlung (engl. exception handling) als ein Mittel, um Fehlerbedingungen zu testen. Dies ist so weit in Python integriert, dass es teilweise sogar möglich ist, Syntaxfehler abzufangen und zur Laufzeit zu behandeln.\\
\\
Ausnahmen haben einige Vorteile gegenüber anderen beim Programmieren üblichen Verfahren der Fehlerbehandlung (wie z. B. Fehler-Rückgabewerte und globale Statusvariablen). Sie sind Thread-sicher und können leicht bis in die höchste Programmebene weitergegeben oder an einer beliebigen anderen Ebene der Funktionsaufruffolge behandelt werden. Der korrekte Einsatz von Ausnahmebehandlungen beim Zugriff auf dynamische Ressourcen erleichtert es zudem, bestimmte auf Race Conditions basierende Sicherheitslücken zu vermeiden, die entstehen können, wenn Zugriffe auf bereits veralteten Statusabfragen basieren.\\
\\
Der Python-Ansatz legt den Einsatz von Ausnahmen nahe, wann immer eine Fehlerbedingung entstehen könnte. Nützlich ist dieses Prinzip beispielsweise bei der Konstruktion robuster Eingabeaufforderungen:\\
\\
\displaycode{\\
while True:\\
\blank{1cm}num = raw\_input("Eine Zahl eingeben: ")\\
\blank{1cm}try:\\
\blank{1cm}\blank{1cm}num = int(num)\\
\blank{1cm}\blank{1cm}break\\
\blank{1cm}except ValueError:\\
\blank{1cm}\blank{1cm}print("Eine Zahl bitte!")\\
}\\
\\
Dieser Code wird den Benutzer so lange nach einer Nummer fragen, bis dieser eine Zeichenfolge eingibt, die sich per int() in eine Ganzzahl konvertieren lässt. Durch die Ausnahmebehandlung wird hier vermieden, dass eine Fehleingabe zu einem Laufzeitfehler führt, der das Programm zur Beendigung zwingt.\footcite{python_wiki}

\subsection{C}\displayauthor{Michael Friesenhengst}\ \\
C entstand Anfang der siebziger Jahre im Zusammenhang mit dem Betriebssystem Unix an den AT\&T Bell Laboratories. Eine wesentliche Rolle spielte dabei Dennis Ritchie.\\
Wichtige Ideen stammen aus der Sprache BCPL, die von M. Richards entwickelt wurde.\\
Unmittelbarer Vorgänger von C ist das von K. Thompson 1970 im Zusammenhang mit der Unix-Entwicklung entstandene B.\\
\\
Bekannt wurde C vor allem im Zusammenhang mit dem Ende der siebziger Jahre freigegebenen Betriebssystem Unix, Version 7, und dem 1978 erschienenen Buch ''The C Programming Language'' von Kernighan und Ritchie bzw. durch den Report Programming in C: A Tutorial von Kernighan.\\
\\
Mitunter wird C etwas grob vereinfachend als eine Kreuzung von Pascal und Assembler bezeichnet:\\

C unterstützt in ähnlichem Umfang wie Pascal strukturiertes Programmieren.
Die Syntax von C und Pascal folgt - bei unterschiedlicher Ausprägung - teilweise gleichen Grundideen. (Diese gehen auf die gemeinsame Vorgängersprache ALGOL 60 zurück. Beeinflusst hat beide Sprachen auch die Diskussion um ALGOL 68.)\\
C gestattet - fast in gleichem Maße wie Assembler - maschinennahes Programmieren.\\
Ähnlich wie Assembler lässt C dem Programmierer einen sehr weitgehenden Gestaltungsspielraum. \\

Im Gegensatz zu den ersten höheren Programmiersprachen wie FORTRAN, ALGOL 60 (wissenschaftlich-technische Berechnungen) und COBOL (kommerzielle Anwendungen) ist C ''universell'' einsetzbar. Gegenüber in den sechziger Jahren entstandenen Universalsprachen wie PL/I oder ALGOL 68, die letztendlich an ihrer Komplexität scheiterten, besitzt C einen eng begrenzten Sprachumfang.\\
Dieser eng begrenzte Sprachumfang hat vor allem zur schnellen Portierung von C auf alle gängigen Rechnerplattformen beigetragen.\\
Der C-Sprachumfang lässt jedoch eine ungeheure Flexibilität zu, so dass auch ''Programmentwicklung in C'' zu einem sehr komplexen Thema wird.\\
Im Vergleich zum etwa zur gleichen Zeit entstandenen Pascal ist C weit besser für die professionelle Software-Entwicklung einsetzbar, ist jedoch als Ausbildungssprache wenig geeignet.\\
\\
C lässt bewusst Spielraum zwischen zwei Polen:
\begin{itemize}
\item Entwicklung hochportabler Programme
\item Einsatz als ''höherer Assembler''\\
d.h. Entwicklung maschinennaher, nicht portabler Programme 
\end{itemize}\cite{c_lang_intro}

\subsubsection{Verwendung}
Trotz des eher hohen Alters ist die Sprache C auch heute weit verbreitet und wird sowohl im Hochschulbereich, wie auch in der Industrie und im Open-Source-Bereich verwendet.\footcite{c_wiki}
\paragraph{System- und Anwendungsprogrammierung}
\ \\
Das Haupteinsatzgebiet von C liegt in der Systemprogrammierung einschließlich der Erstellung von Betriebssystemen und der Programmierung von eingebetteten Systemen. Der Grund liegt in der Kombination von erwünschten Charakteristiken wie Portabilität und Effizienz mit der Möglichkeit, Hardware direkt anzusprechen und dabei niedrige Anforderungen an die Laufzeitumgebung zu haben.\\
\\
Auch Anwendungssoftware wird oft in C erstellt. Viele Programmierschnittstellen für Anwendungsprogramme und Betriebssystem-APIs werden in Form von C-Schnittstellen implementiert, zum Beispiel Win32. Gemäß C-Standard existieren jedoch keine Funktionen zur positionierten Ausgabe auf Displays, d.h. text- oder grafisch orientierte Benutzeroberflächen sind in reinem C nicht realisierbar. Es existieren jedoch zahlreiche Bibliotheken, die für das jeweilige Zielsystem eine solche Ausgabe ermöglichen.\footcite{c_wiki}\\

\paragraph{Implementierung anderer Sprachen}
\ \\
Wegen der hohen Ausführungsgeschwindigkeit und geringen Codegröße werden Compiler, Programmbibliotheken und Interpreter anderer höherer Programmiersprachen (wie z. B. die Java Virtual Machine) oft in C implementiert.\\
\\
C wird als Zwischencode einiger Implementierungen höherer Programmiersprachen verwendet. Dabei wird diese zuerst in C-Code übersetzt, der dann kompiliert wird. Dieser Ansatz wird entweder verwendet, um die Portabilität zu erhöhen (C-Compiler existieren für nahezu jede Plattform), oder aus Bequemlichkeit, da kein maschinenspezifischer Codegenerator entwickelt werden muss.\\
\\
C wurde allerdings als Programmiersprache und nicht als Zielsprache für Compiler entworfen. Als Zwischensprache ist es daher eher schlecht geeignet. Das führte zu C-basierten Zwischensprachen wie C--.\\
\\
C wird oft für die Erstellung von Anbindungen (engl. bindings) genutzt (zum Beispiel Java Native Interface). Diese Anbindungen erlauben es Programmen, die in einer anderen Hochsprache geschrieben sind, Funktionen aufzurufen, die in C implementiert wurden. Der umgekehrte Weg ist oft ebenfalls möglich und kann verwendet werden, um in C geschriebene Programme mit einer anderen Sprache zu erweitern.\footcite{c_wiki}\\

\subsubsection{Syntax}

C besitzt eine sehr kleine Menge an Schlüsselwörtern. Die Anzahl der Schlüsselwörter ist so gering, weil fast alle Aufgaben, welche in anderen Sprachen über eigene Schlüsselwörter realisiert werden, über Funktionen der C-Standard-Bibliothek realisiert werden (zum Beispiel die Ein- und Ausgabe über Konsole oder Dateien, dynamische Speicherverwaltung, usw.).\footcite{c_wiki}\\

\paragraph{Datentypen}
\ \\
\ \\
\textbf{char}\\
Zum Speichern eines Zeichens (sowie von kleinen Zahlen) verwendet man in C üblicherweise den Integer-Datentyp Character, geschrieben als \displaycode{char}. Vom Computer tatsächlich gespeichert wird nicht das Zeichen (wie zum Beispiel ''A'') sondern eine gleichbedeutende acht Bit lange Binärzahl (wie zum Beispiel 01000001). Diese Binärzahl steht im Speicher und kann anhand einer Tabelle jederzeit automatisch in den entsprechenden Buchstaben umgewandelt werden. Zum Beispiel steht 01000001 gemäß der ASCII-Tabelle für das Zeichen ''A''. Um auch Zeichen aus Zeichensätzen aufnehmen zu können, die mehr Zeichen umfassen als der relativ kleine ASCII-Zeichensatz, wurde mit wchar\_t bald ein zweiter für Zeichen konzipierter Datentyp eingeführt.\\
\ \\
\displaycode{
char zeichen = 'A';    /* gespeichert wird 01000001 */\\
printf(''\%d'', zeichen); /* gibt 01000001 als Dezimalzahl aus (65) */\\
printf(''\%c'', zeichen); /* gibt 01000001 als ASCII-Zeichen aus (''A'') */
}\\
\ \\
\textbf{int}\\
Zum Speichern einer Ganzzahl (wie zum Beispiel 3) verwendet man eine Variable vom Datentyp Integer, geschrieben als \displaycode{int}. Die Größe eines Integers beträgt heutzutage (je nach Prozessorarchitektur und Betriebssystem) meist 32 Bit, oft aber auch schon 64 und manchmal noch 16 Bit. In 16 Bit lassen sich 65536 verschiedene Werte speichern. Um die Verwendung von negativen Zahlen zu ermöglichen, reicht der Wertebereich bei 16 Bit gewöhnlich von -32768 bis 32767. Werden keine negativen Zahlen benötigt, kann der Programmierer mit \displaycode{unsigned int} aber einen vorzeichenlosen Integer verwenden. Bei 16 Bit großen Integern ergibt das einen Wertebereich von 0 bis 65535.\\
\\
Um den Wertebereich eines Integers zu verkleinern oder zu vergrößern, stellt man ihm einen der Qualifizierer short, long oder long long voran. Das Schlüsselwort \displaycode{int} kann dann auch weggelassen werden, so ist \displaycode{long} gleichbedeutend mit \displaycode{long int}. Um zwischen vorzeichenbehafteten und vorzeichenlosen Ganzzahlen zu wechseln, gibt es die beiden Qualifizierer \displaycode{signed} und \displaycode{unsigned}. Für einen vorzeichenbehafteten Integer kann der Qualifizierer aber auch weggelassen werden, so ist \displaycode{signed int} gleichbedeutend mit \displaycode{int}. Die C-Standard-Bibliothek ergänzt diese Datentypen über die plattformunabhängige Header-Datei <stdint.h> in der ein Set von Ganzzahltypen mit fester Länge definiert ist.\\
\ \\
\displaycode{
char ganzzahl = 1;      /* >= 8b, 256 mögliche Werte */\\
short ganzzahl = 2;     /* >= 16b, 65536 mögliche Werte */\\
int ganzzahl = 3;       /* >= 16b, 65536 mögliche Werte */\\
long ganzzahl = 4;      /* >= 32b, 4294967296 mögliche Werte */\\
long long ganzzahl = 5; /* >= 64b, 18446744073709551616 mögliche Werte */\\
}\\
\ \\
\textbf{float und double}\\
Zahlen mit Nachkommastellen werden in einem der drei Datentypen \displaycode{float}, \displaycode{double} und \displaycode{long double} gespeichert. In den meisten C-Implementierungen entsprechen die Datentypen Float und Double dem international gültigen Standard für binäre Gleitpunktarithmetiken (IEC 559, im Jahr 1989 aus dem älteren amerikanischen Standard IEEE 754 hervorgegangen). Ein Float implementiert das „einfach lange Format“, ein Double das „doppelt lange Format“. Dabei umfasst ein Float 32 Bit, ein Double 64 Bit. Doubles sind also genauer. Floats werden aufgrund dieses Umstands nur noch in speziellen Fällen verwendet. Die Größe von Long Doubles ist je nach Implementierung unterschiedlich, ein Long Double darf aber auf keinen Fall kleiner sein als ein Double. Die genauen Eigenschaften und Wertebereiche auf der benutzten Architektur können über die Headerdatei <float.h> ermittelt werden.\\
\ \\
\displaycode{
float kommaz = 0.000001;           /* 6-stellige Genauigkeit */\\
double kommaz = 0.000000000000002; /* 15-stellige Genauigkeit */\\
long double kommaz = 0.3;          /* Genauigkeit ist implementierunsabhängig */\\
}\\
\ \\
\textbf{void}\\
Der Datentyp void wird im C-Standard als ''unvollständiger Typ'' bezeichnet. Man kann keine Variablen von diesem Typ erzeugen. Verwendet wird void erstens, wenn eine Funktion keinen Wert zurückgeben soll und zweitens, wenn ein Zeiger auf ''Objekte beliebigen Typs'' zeigen soll.\\
\ \\
\displaycode{
void funktionsname();      /* Funktion, die keinen Wert zurückgibt */\\
void* zeigername;          /* Zeiger auf ein Objekt von beliebigem Typ */
}\\
\ \\
\textbf{Zeiger}\\
Wie Zeiger in anderen Programmiersprachen sind Zeiger in C Variablen, die statt eines direkt verwendbaren Wertes (wie das Zeichen ''A'' oder die Zahl 5) eine Speicheradresse (wie etwa die Adresse 170234) speichern. die Adressen im Speicher sind durchnummeriert. An der Speicheradresse 170234 könnte zum Beispiel der Wert 00000001 gespeichert sein (Binärwert der Dezimalzahl 1). Zeiger ermöglichen es, auf den Wert zuzugreifen, der an einer Speicheradresse liegt. Dieser Wert kann wiederum eine Adresse sein, die auf eine weitere Speicheradresse zeigt. Bei der Deklaration eines Zeigers wird zuerst der Datentyp des Objekts angegeben, auf das gezeigt wird, danach ein Asterisk, danach der gewünschte Name des Zeigers.\\
\ \\
\displaycode{
char *zeiger;    /* kann die Adresse eines Characters speichern */\\
double *zeiger;  /* kann die Adresse eines Doubles speichern */
}\\
\ \\
\textbf{struct}\\
Um verschiedenartige Daten in einer Variable zu speichern, verwendet man Structures, geschrieben als \displaycode{struct}. Auf diese Weise können Variablen verschiedenen Datentyps zusammengefasst werden.\\
\ \\
\displaycode{
struct person\{\\
\blank{1cm}	char* vorname;\\
\blank{1cm}	char nachname[20];\\
\blank{1cm}	int alter;\\
\blank{1cm}	double groesse;\\
\};
}\\
\ \\
\textbf{enum}\\
Wie in anderen Programmiersprachen dient ein Enum in C dazu, mehrere konstante Werte zu einem Typ zu kombinieren.\\
\ \\
\displaycode{
enum Temperatur \{ WARM, KALT, MITTEL \};\\
enum Temperatur heutige\_temperatur = WARM;\\
if(heutige\_temperatur == KALT)\{\\
\blank{1cm}printf(''Warm anziehen!'');               /* keine Ausgabe, da es ''WARM'' ist */\\
\}
}\\
\ \\
\textbf{typedef}\\
Das Schlüsselwort Typedef wird zur Erstellung eines Alias für einen Datentyp verwendet.\\
\ \\
\displaycode{
typedef int Ganzzahl;  /* legt Alias ''Ganzzahl'' für ''int'' an */\\
Ganzzahl a, b;         /* ist jetzt gleichbedeutend zu ''int a, b;'' */
}\\
\ \\
\textbf{\_Bool}\\
Bis zum C99-Standard gab es keinen Datentyp zum Speichern eines Wahrheitswerts. Erst seit 1999 können Variablen als \_Bool deklariert werden und einen der beiden Werte 0 (falsch) oder 1 (wahr) aufnehmen. Inkludiert man den Header stdbool.h kann auch der Alias \displaycode{bool} statt \displaycode{\_Bool} verwendet werden, sowie \displaycode{false} und \displaycode{true} statt \displaycode{0} und \displaycode{1}.\\
\ \\
\displaycode{
\_Bool a = 1;   /* seit C99 */
}\\
\ \\
\textbf{\_Complex und \_Imaginary}\\
Seit C99 gibt es drei Gleitkomma-Datentypen für komplexe Zahlen, welche aus den drei Gleitkommatypen abgeleitet sind: float \_Complex, double \_Complex und long double \_Complex. Ebenfalls in C99 eingeführt wurden Gleitkomma-Datentypen für rein imaginäre Zahlen: float \_Imaginary, double \_Imaginary und long double \_Imaginary.\\
\ \\
\cite{c_wiki}
\subsubsection{Funktionen}

Ein C-Programm besteht aus der main-Funktion und optional aus weiteren Funktionen. Weitere Funktionen können entweder selbst definiert werden oder vorgefertigt aus der C-Standard-Bibliothek übernommen werden.\footcite{c_wiki}

\paragraph{main}
\ \\
Jedes C-Programm muss eine Funktion mit dem Namen main haben, anderenfalls wird das Programm nicht kompiliert. Die main-Funktion ist der Einsprungspunkt eines C-Programms, das heißt die Programmausführung beginnt immer mit dieser Funktion.\\
\\
Außer der main-Funktion müssen in einem C-Programm keine weiteren Funktionen enthalten sein. Sollen andere Funktionen ausgeführt werden, müssen sie in der main-Funktionen aufgerufen werden. Die main-Funktion wird deshalb auch als Hauptprogramm bezeichnet, alle weiteren Funktionen als Unterprogramme.\footcite{c_wiki}
\paragraph{Selbstdefinierte Funktionen}
\ \\
In C lassen sich beliebig viele Funktionen selbst definieren. Eine Funktionsdefinition besteht aus erstens aus dem Datentyp des Rückgabewerts, zweitens dem Namen der Funktion, drittens einer eingeklammerten Liste von Parametern und viertens aus einem eingeklammerten Funktionsrumpf, in welchem ausprogrammiert wird, was die Funktion tun soll.\\
\ \\
\displaycode{
int summe (int x, int y)\{  /* Datentyp, Funktionsname, zwei Parameter */\\
\blank{1cm}	return x + y;         /* Funktionsrumpf (Summenberechnung)*/\\
\}\\
\\
int main (void)\{\\
\blank{1cm}		int ergebnis = summe(2, 3);   /* Funktionsaufruf mit den Werten 2 und 3, Rückgabewert wird in „ergebnis“ gespeichert */\\
\blank{1cm}	return ergebnis;              /* main gibt den Wert von „ergebnis“ zurück */\\
\}
}\\
\ \\
Für die Definition einer Funktion, die nichts zurückgeben soll, verwendet man das Schlüsselwort void. Ebenso falls der Funktion keine Parameter übergeben werden sollen.\\
\ \\
\displaycode{
void begruessung (void)\{\\
\blank{1cm}		printf("Hi!");\\
\blank{1cm}		return;\\
\}
}

\cite{c_wiki}

\paragraph{C-Standard-Bibliothek}
\ \\
Die Funktionen der Standard-Bibliothek sind nicht Teil der Programmiersprache C. Sie werden jedoch mit fast jedem Compiler mitgeliefert und können verwendet werden, sobald man die jeweils entsprechende Header-Datei eingebunden hat. Beispielsweise dient die Funktion printf zur Ausgabe von Text. Sie kann verwendet werden, nachdem man die Header-Datei stdio.h eingebunden hat.\\
\ \\
\displaycode{
\#include <stdio.h>;\\
main()\{\\
\blank{1cm}	printf("hello world!\\n");\\
\}
}
\ \\
\cite{c_wiki}
\ \\
\paragraph{Anweisungen}
\ \\
Eine Funktion besteht aus Anweisungen. Wie in den meisten Programmiersprachen sind die wichtigsten Anweisungen: Deklarationen und Definitionen, Zuweisungen, bedingte Anweisungen, Anweisungen die Schleifen umsetzen sowie Funktionsaufrufe.\\
\ \\
\displaycode{
void funktion\_die\_nichts\_tut(void)\{    /* Definition */\\
\blank{1cm}	return;                             /* Return-Anweisung */\\
\}\\
int plus\_eins\_funktion(int argument)\{  /* Definition */\\
\blank{1cm}	return argument + 1;                /* Return-Anweisung */\\
\}\\
\\
int main()\{                            /* Definition */\\
\blank{1cm}	int zahl;                           /* Definition */\\
\blank{1cm}	funktion\_die\_nichts\_tut();          /* Funktionsaufruf */\\
\blank{1cm}	zahl = 5;                           /* Zuweisung */\\
\blank{1cm}	zahl = plus\_eins\_funktion(zahl);    /* Funktionsaufruf und Zuweisung */\\
\blank{1cm}	if(zahl > 5)\{                      /* bedingte Anweisung */\\
\blank{1cm}\blank{1cm}		zahl = zahl - 1;                /* Zuweisung: Wert von „zahl“ ist wieder „5“ */\\
\blank{1cm}	\}\\
\blank{1cm}	return 0;                           /* Return-Anweisung */\\
\}
}
\ \\
\cite{c_wiki}

\subsubsection{Namen}
Beim Benennen von eigenen Variablen, Konstanten, Funktionen und Datentypen muss man sich an einige Regeln zur Namensgebung halten. Erstens muss das erste Zeichen eines Bezeichners ein Buchstabe oder Unterstrich sein. Zweitens dürfen die folgenden Zeichen nur die Buchstaben A bis Z und a bis z, Ziffern und der Unterstrich sein. Und drittens darf der Name keines der Schlüsselwörter sein.\\
\\
Seit C95 sind auch Zeichen aus dem Universal Character Set in Bezeichnern erlaubt, sofern die Implementierung es unterstützt. Die erlaubten Zeichen sind in Anhang D des ISO-C-Standards aufgelistet. Vereinfacht gesagt, sind es all jene Zeichen, die in irgendeiner Sprache als Buchstabe oder buchstabenähnliches Zeichen Verwendung finden. Im Quelltext lassen sich diese Zeichen plattformunabhängig über eine Escape-Sequenz wie folgt ersetzen:
\begin{itemize}
\item \textbackslash uXXXX (wobei X für eine Hexadezimalziffer steht) für Zeichen mit einem Code von 00A0 bis FFFF.
\item \textbackslash UXXXXXXXX für alle Zeichen mit einem Code >= 00A0.
\end{itemize}
Bestimmte Bezeichner sind außerdem für die Implementierung reserviert:
\begin{itemize}
\item Bezeichner, die mit zwei aufeinanderfolgenden Unterstrichen beginnen
\item Bezeichner, die mit Unterstrich gefolgt von einem Großbuchstaben anfangen.
\end{itemize}
Erweiterungen am Sprachkern, die neue Schlüsselwörter erfordern, verwenden dafür ebenfalls Namen aus diesem reservierten Bereich, um zu vermeiden, dass sie mit Bezeichnern in existierenden C-Programmen kollidieren, z. B. \displaycode{\_\_attribute\_\_}, \displaycode{\_Complex}, \displaycode{\_Generic}.\footcite{c_wiki}

\section{Hardware}
Um eine volle Alternative zu jeglichen Taschenrechnern bieten zu können, musste eine entsprechende Hardware entwickelt werden, auf der man problemlos unsere Software bedienen kann. Dazu brauchten wir eine Recheneinheit, einen Display, eine für den Benutzer einfach zu bedienende Eingabemöglichkeit und ein passendes Gehäuse.\\

Von Anfang an war klar, dass die Hauptrecheneinheit unseres Prototypen fähig sein musste, den Rechenaufwand unserer Software problemlos zu bewältigen. Ein weiteres wichtiges Kriterium war, einen kleinen Formfarktor einhalten zu können, um zu garantieren, dass unsere Einheit in einem handlichen Gehäuse Platz findet. Beiden Punkten zugleich wird lediglich ein \textit{single-board-computer} gerecht, er vereint hohe Rechenleistung mit minimalem Platzaufkommen.\\

Das Display musste groß genug sein, um dem Benutzer genug Platz zu bieten, seine Berechnungen und Notizen durchzuführen. Allerdings war auch wichtig darauf zu achten, dass es nicht zu groß war um nicht die Portabilität unseres Gerätes einzuschränken.\\

Mittels dem \textit{Touchscreen} ist es möglich, Berechnungen einzugeben sowie Mitschriften zu tätigen. Er wird mithilfe eines Stiftes bedient.\\

Da der single-board-computer und Display einen relativ hohen Energiebedarf aufweisen, musste eine Hardware-Komponente her, welche beide je nach Bedarf aus-und-einschalten kann. Bei dieser Komponente war darauf zu achten, dass diese dem Gerät so wenig Energie wie möglich entzieht um eine lange \textit{Standby-Zeit} zu garantieren. Die Komponente kann auf Knopfdruck Display und Recheneinheit getrennt von einander stromlos machen.\\

\subsection{Recheneinheit}
Bei der großen Auswahl an single-board-computern entschieden wir und für den \textit{Raspberry Pi 3}, da dieser neben hervorragendem Treibersupport, den wir für unseren Touchscreen benötigen würden, durch eine für den Formfaktor sehr hohe Leistung überzeugt. Weiters ist bereits eine \textit{Wireless-LAN} Schnittstelle verbaut, welche bei der Software-Installation sehr behilflich ist.\\



\subsection{Display}



\subsection{Touchscreen}

Da es bei Touchscreens viele verschiedene Technologien gibt, welche heutzutage eingesetzt werden, war es nicht einfach einen passenden für unser Display-Modell zu finden.\\

Unbedingt notwendig war, dass es sich um einen \textit{resistiven} Touchscreen handelt welcher mittels Druck funktioniert. Dadurch ist es möglich den Touchscreen mit einem Stift zu bedienen. Dies ist bei einem heutzutage üblichen \textit{kapazitiven} Touchscreen nicht möglich, da dieser nicht auf Druck sondern auf die Änderung der Kapazität reagiert.\\

Wir entschieden uns für einen Touchscreen der Marke \textit{EETI}, der mittels USB an den Raspberry Pi angeschlossen werden kann, da für diesen offizieller Treibersupport zur Verfügung steht.\\

\subsection{Power-Button}

Es war uns sehr wichtig, den Display und den Raspberry Pi durch einfaches Klicken eines Knopfes vom Strom trennen zu können, dies schränkt nicht nur den Stromverbrauch des Gerätes drastisch ein, es macht es auch möglich, den Display und Touchscreen unabhängig vom Raspberry Pi abzuschalten.\\

Wir setzten uns folgende Anforderungen für unseren Knopf:\\
\textit{Durch einfaches, kurzes Klicken auf den Knopf wird der Display ein- bzw. ausgeschaltet.
Durch längeres, dreisekündiges Halten des Knopfes und anschließendes Loslassen, wird die Recheneinheit ein- bzw. ausgeschaltet.}

Um diese Anforderungen umzusetzen nahmen wir einen \textit{Mikrocontroller} zur Hilfe. Wir entschieden uns für einen \textit{Arduino Pro Mini}, da dieser einen minimalen Energieverbrauch hat und leicht mit der Programmiersprache \textit{C} programmierbar ist.\\

Da es nicht möglich ist, sehr große Ströme mit dem Arduino allein zu schalten, verbauten wir für Display und Recheneinheit jeweils einen \textit{Leistungs-NFET}, der einen Strom von bis zu 3 Ampere schalten kann, ausreichend für unsere Zwecke also.\\

Weiters musste es möglich sein, dem Raspberry Pi ein Ausschaltsignal zu senden, welches ihn dazu bringt, sich selbst herunterzufahren um Korruption des Raspbian-Systems auf der Recheneinheit zu verhindern. Auch in die andere Richtung musste eine Signalleitung gelegt werden, die dem Mikrokontroller anzeigt, wenn der Raspberry Pi vollständig heruntergefahren ist, um ihn anschließend stromlos zu machen. Dazu wurde in jede Richtung jeweils ein \textit{NPN-Transistor} verbaut, um eine Brücke zwischen den 5 Volt des Arduino und den 3,3 Volt des Raspberry Pi herzustellen.\\

Um einen minimalen Leistungsaufwand des Mikrocontrollers zu garantieren wurde die LowPower-sssLibrary verwendet und mit Interrupts programmiert.

\subsection{Akku}

\subsection{Gehäuse}


\section{Software}
Besonders, wenn es um große Projekte wie dieses geht, ist Software tatsächlich kein einmaliges Produkt, sondern etwas, was sich über den Lauf des Projektes und darüber hinaus verändert. Aus diesem Grund ist es unserer Meinung nach äußerst wichtig, nur Lösungen zu publizieren, die zukunftstauglich und offen für Erweiterungen sind. Im folgenden Kapitel möchten wir darauf eingehen, wie diese Denkweise in unserer Arbeit umgesetzt wurde und warum davon auch langfristig profitiert werden kann.\\
\\
Als wir begannen, uns über die Software-Strukturen Gedanken zu machen, war es klar, dass es eine Komponente bräuchte, die sich mit der Benutzer-Interaktion beschäftigt, und eine, in der mathematische Prozesse durchlaufen, aber auch abgebildet werden sollten. Es musste also abgewogen werden, ob besser eines eine vorrangige Stellung haben sollte, oder ob im Endeffekt zwei Systeme, die sich komplementieren, am Laufen sein sollten. An den Beispielen existierender Software erkannten wir, dass zwei parallele Systeme, um ideal zu funktionieren, andauernd in Synchronisation gehalten werden müssten. Um diesen doppelten Aufwand zu umgehen, entschieden wir, das Mathematik-Paket in die grafische Oberfläche zu integrieren, da der mathematische Teil eben nur eine der Funktionalitäten unseres Produkts ist, wenn auch eine sehr wichtige. Das Grundgerüst stellt also das \textit{Graphical User Interface} oder kurz \textit{GUI} dar.
\subsection{GUI - Das Graphical User Interface}

%\newpage
\subsection{Das Mathematik-Paket}
\textit{Mathematik-Paket} ist der interne Name für diejenige Software Komponente, die alles in Bezug auf mathematische Vorgänge verarbeitet. Aus dem Versuch, diese Komponente in Programmcode zu fassen, erlangte ich eine wesentliche Erkenntnis: Theoretische Mathematik und Informatik sind viel weiter voneinander entfernt, als man glauben mag.\\\\
In der Mathematik sieht man Konstrukte als abstrakt, soweit sollte die mathematische Sicht der Dinge ja bekannt sein. Durch Definitionen schafft sich ein Mathematiker Freiraum, wodurch auch der Level der Abstraktion steigt. Zum Beispiel ist uns möglicherweise nicht sofort bewusst, dass wir beim 'mit $x$ multiplizieren' tatsächlich $x$-Mal zusammenzählen. Beim 'hoch $x$ rechnen' wird eigentlich $x$-Mal multipliziert, also wie oft addiert? Natürlich stellen die Grundrechenarten programmtechnisch kein Problem dar, jedoch ist bereits diese einfache Frage ohne mehr Information nicht zu beantworten, was zeigen soll, wie tief wir bereits drin stecken, ohne jemals angefangen zu haben.\\\\
In diesem Teil der Programmierung geht es also darum, die abstrakten mathematischen Beschreibungen der Dinge in konkretisierte, ausführbare, Programmstücke zu fassen. Zuerst haben wir uns ein Grundkonzept überlegt, das folgenden Anforderungen entsprechen sollte:
\begin{itemize}
	\item Alle Mathematischen Vorgänge sollen abbildbar sein
	\begin{itemize}
		\item Rechnungen
		\item Gleichungen
		\item Funktionen/Variablendefinitionen
		\item Gleichungssysteme
	\end{itemize}
	\item geringer \textit{RAM}-Aufwand (Arbeitsspeicher)
	\item durch geringen Aufwand änderbar
	\item durch geringen Aufwand durchsuchbar
\end{itemize}
Der Grund für die notwendige Änderungsfähigkeit ist ganz einfach, dass der Benutzer zu jedem Zeitpunkt in der Lage sein soll, die Eingabe zu bearbeiten. Die Suchfunktion ist erklärbar, wenn man einen Schritt weiter denkt: mathematische Funktionen sowie andere Vorgänge müssen unvermeidlich die zu ersetzenden Variablen ausfindig machen und stattdessen Zahlen einfügen. Ich spreche hier "andere Vorgänge" an, da beispielsweise Gleichungssysteme oder das einfache Einsetzen von Variablen diese Funktionalität ebenfalls gebrauchen könnten, das hängt jedoch von der Implementierung ab.\\
Schlussendlich fanden wir zwei Lösungen, die für die engere Auswahl in Frage kamen:\\
\\
Lösung 1: Gesamte Information in einen \textit{String}  verpacken, etwa so, dass $2 \cdot x+3 $ als \textbf{" \textbackslash Sum( \textbackslash Product( \#2, \_x ), \#3)"} gespeichert wird. Ein \textit{String} unterscheidet sich im Wesentlichen nicht von einem Text im Programmier-Jargon (der Ausdruck stammt von "Zeichenkette" ab).\\
\\
Lösung 2: Struktur wird per \textit{Objekt(e)} gespeichert. Jedes Objekt sollte für eine Operation oder etwas Vergleichbares stehen. Hier wäre unser Term $2 \cdot x+3 $, wenn so dargestellt, dass die Eigenschaften eines Objektes darunter aufgelistet sind, Folgendes:\\
\textbf{
\begin{itemize}
	\item[$-$] Sum
	\begin{itemize}
		\item[$-$] Product
		\begin{itemize}
			\item[$-$] Constant: $2.0$
			\item[$-$] Variable: $x$
		\end{itemize}
		\item[$-$] Constant: $3.0$
	\end{itemize}
\end{itemize}
}
\ \\
Variante 1 hatte den wesentlichen Vorteil, dass suchen und ersetzen äußerst einfach gewesen wären, da Java für Strings schon standardmäßig Funktionen hierfür anbietet. Wie man sehen kann, ist die erste Version auch kompakter, dafür kann sie mit Struktur nicht punkten: Wo ein bestimmtes Element beginnt und endet ist nirgends hinterlegt, dies dürfte also bei jedem Durchgang aufs Neue ermittelt werden. Das entscheidende Argument für die Objekt-Struktur war letztendlich, dass die Modularität viel präziser ablaufen könnte. Das kann man sich so vorstellen, dass, um Strings zu verarbeiten sinnvollerweise eine zentrale Stelle (gemeint: \textit{Methode}) für die Interpretation verantwortlich ist. Das bedeutet, dass sich sämtliche Erweiterungen an die Regeln dieser Stelle halten müssten. Bei Variante 2 hingegen wären sämtliche Funktionen ausgelagert, sprich: jedes Objekt arbeitet nach Belangen, solange globale Anforderungen erfüllt werden. Mit "Anforderungen" ist hier einfach gemeint, dass es berechenbar ist, in der Lage ist, Variablen zu ersetzen usw.\\
\subsubsection{Term-Struktur}
Die Wahl fällt also eindeutig auf die oben angeführte Lösung Nr. 2, sie ist flexibler, strukturierter und modularer als ihre Alternative. In der Code-Implementierung bietet es sich an, sich die \textit{Vererbung} zu Nutze zu machen. Das bedeutet in Java, dass eine \textit{Mutterklasse} (auch: \textit{Überklasse}) existiert, die so etwas wie ein Gerüst vorgibt. Ich verwende hier bewusst nicht den Ausdruck "Grundgerüst", da auch Mutterklassen weitere Mutterklassen haben können, wobei jede Ebene zum letzten Objekt der Kette beiträgt. Solch eine Überklasse muss einfach mit dem Schlüsselwort \displaycode{abstract} gekennzeichnet werden und darf praktisch die Umrisse für alle Methoden enthalten, obwohl sie keine einzige genau beschreiben muss. Das hat den Sinn, dass Objekte klassifiziert werden können und trotzdem ihre Funktionalität eigenständig verwalten dürfen.\\
\\
Im konkreten Fall bedeutet dies nun, dass wir eine abstrakte Mutterklasse \displaycode{Term} geschrieben haben, die - sehr unspektakulär - mathematische Terme (Ausdrücke) darstellen soll. Ein Term ist alles von Summe über Bruch, Logarithmus bis zur einfachen Konstante. Hinter der \displaycode{Constant} ($\rightarrow$ Konstante) verbirgt sich nichts anderes als eine Zahl, wir dachten jedoch, der Name würde besser im Kontrast zur \displaycode{Variable} ($\rightarrow$ Variable) stehen. Diese Klasse legt nun einige abstrakte Methoden fest, wie zum Beispiel \displaycode{public Term calc()}, in der das jeweilige Unterobjekt seinen Inhalt berechnen und zurückgeben soll. Die gleiche Methode gibt es auch mit Parameter, nämlich einem Objekt, das alle aktuellen Variablenwerte hält, was eben für Funktionen wichtig wäre. Als kurze Veranschaulichung würde diese Methode zum Beispiel bei der \displaycode{Sum} ($\rightarrow$ Summe) ganz einfach so aussehen, dass sie alle ihre Summanden auffordert, deren Inhalt zu berechnen und diese im Anschluss addiert.\\
\\
Die nächste Anforderung war, dass sie neben einigen Methoden, die ich bis jetzt noch nicht erwähnt habe, wie \displaycode{void simplify()} zum vereinfachen oder \displaycode{void changeSubterm(Term o, Term n)} aus strukturtechnischen Gründen, zusätzlich eine Funktionalität zum Ableiten nach einer Variable besitzen. Ich werde nun die Implementation der Methode \displaycode{Term derive()} in der Klasse \displaycode{Sum} demonstrieren. In der Arbeit wurde dies natürlich für alle Unterklassen fertiggestellt. Hier nun die Ableitungsregel für Summen:
\begin{gather*}
\frac{d(f(x)+g(x)}{dx} = \frac{d f(x)}{dx}+\frac{d g(x)}{dx}
\end{gather*}
Der Java-Code liest sich folgendermaßen:\\

\begin{center}
\includegraphics[width=0.9\textwidth]{img/sachsenschnitzel/code_Sum_derive_cropped}
\end{center}
\noindent
Ich denke an diesem Beispiel kann man äußerst klar erkennen, dass die Wahl unserer derzeitigen Struktur zu einigen guten Ergebnissen geführt hat. Der Programmcode ist gekapselt, modular und einfach lesbar.
%klammern? nicht-wirklich-terme?

\subsubsection{Terme, Funktionen, Gleichung(ssysteme)}
Die Terme wären hiermit soweit abgedeckt, was allerdings tun mit Gleichungen? Sind Gleichungen nicht eigentlich auch Terme?\\
Beim weiteren Nachdenken fällt auf, dass Terme der Bedingung gehorchen, dass man sie schachteln kann, so kann eine Summe beispielsweise weitere Summen, genauso aber Konstanten oder jede andere Art von Term beinhalten. Zusammengefasst sind ihre Summanden also einfach weitere Terme, die beliebig tief geschachtelt sein dürfen. Unsere anfängliche Frage sollte nun einfach zu klären sein: kann eine Summe eine Gleichung beinhalten? Nein. Auf diese Antwort hätte man genauso in der Mathematik stoßen können: Laut Definition ist eine Gleichung eine \citebrackets{Aussage über die Gleichheit zweier Terme} und \citebrackets{entweder wahr oder falsch}\footcite{math_gleichung} und ist damit selbst kein Term, da in der Mathematik niemand mit einer wahr/falsch-Information weiter rechnen kann.\\
\\
In diesem Sinne wurden Gleichungen logischerweise implementiert, sie besitzen zwei Terme, über die sie verfügen können. Aber was nun ist mit Funktionen? Funktionen scheinen berechenbar. Allerdings sollte man hier vorsichtig sein, denn streng genommen kann man, ähnlich wie in der Informatik, zwischen zweierlei Verwendung der Funktionen unterscheiden: Auf der einen Seite spricht man von der Definition der Funktion, der Teil, bei dem man etwa $f(x) = 2 \cdot x  - 4$ schreibt, um anzuzeigen, was in der Funktion passieren soll. Aber wann soll das passieren? Damit wären wir bei der anderen Anwendung, wo eine Funktion verwendet oder \textit{aufgerufen} wird. Hier könnte zum Beispiel $f(x=3)$ das Ergebnis $2 \cdot 3 - 4 = 2$ liefern. Im Endeffekt werden im Programmcode also beide Varianten benötigt, von welchen eine ein \displaycode{Term} und die andere kein solcher sein soll. Die Funktionsdefinition sieht sogar eher einer Gleichung ähnlich, im Endeffekt wird sie immerhin dargestellt als $<Funktionsname>(<Funktionsvariable(n)>)=<Funktionsterm>$.\\
\\
Da Gleichungen ja bereits behandelt wurden, ist es ziemlich offensichtlich, dass Gleichungssysteme ebenfalls keine Untergruppe der Terme darstellen. Sie werden ganz einfach mehrere Gleichungen beinhalten und eine Funktionalität zum Auslesen der \textit{Koeffizienten} (=Zahlen vor einzelnen Variablen) besitzen, was später noch von Bedeutung sein wird.\\
\\
Unserer Meinung nach ist es sinnvoll, auch für diese mathematischen Objekte einen gemeinsamen Nenner zu finden. Warum wir so entschieden haben, wird sich in weiteren Kapiteln noch zeigen. Also haben wir kurzerhand die Klasse \displaycode{MathObject} ins Leben gerufen. Sie soll, wie der Name bereits verrät, als Über-Objekt für alle mathematischen Konstrukte wirken, sodass \displaycode{Term}, \displaycode{Function}, \displaycode{Equation} und \displaycode{SystemOfEquations} \textit{Child-Objekt} davon sind.

\section{Lösung linearer Gleichungssysteme in Java}
Da nun Objektstruktur und Vererbung geklärt sind, besteht der nächste Schritt darin, Berechnungen durchzuführen. Als Einstieg in das Thema möchte ich mit Gleichungssystemen beginnen. \textit{Lineare Gleichungssysteme} zu lösen haben die meisten sicher in der Schule gelernt. Das sind zur Erinnerung jene, bei denen lediglich konstante Koeffizienten vor den Variablen stehen, womit Ausdrücke wie $ln(x)$ oder $x^2$ ausgeschlossen sind. Damit das hier ein wenig anschaulicher wird, werde ich für das gesamte Kapitel folgendes Beispiel-Gleichungssystem verwenden:\\
\begin{gather*}
2x + 4y - z = 17\\
x + 3y + 2z = 18\\
-7x + 5y + z = -29
\end{gather*}
\noindent
Nun gibt es verschiedene Möglichkeiten, Gleichungssysteme zu lösen. Einsetz- oder Substitutionsverfahren, Additionsverfahren, Matrixverfahren sind vielleicht gängige Namen, über die man in diesem Bereich stolpern könnte. Am besten eignet sich für den Computer natürlich eines, bei dem möglichst viel mit Zahlen allein gemacht werden kann. Da das Substitutionsverfahren darauf beruht, dass einzelne Variablen durch andere dargestellt werden können, bis bloß eine übrig ist, scheidet dieses aus. Beim Matrixverfahren wird die Inverse von Matrizen benötigt, deren Berechnung wir als aufwendiger erachteten als das gaußsche Eliminationsverfahren, in dem einfach eine Gleichung nach der anderen gekürzt wird, bis alles von unten aufgelöst werden kann. Dieses Verfahren werde ich nun demonstrieren und anschließend den Programmcode zeigen.\\
Deutlich aufgelistet sieht unser Gleichungssystem folgendermaßen aus:

\[ %Ausgangssystem
\systeme{
	2x  +  4y  -   z  =  17,
	 x  +  3y  +  2z  =  18,
	-7x +  5y  +   z  = -29}
\]

\[ %auf gleich multiplizieren #1
\systeme{
	2x  +  4y  -   z  =  17,
	2x  +  6y  +  4z  =  36,
	2x - \frac{10}{7}y - \frac{2}{7}z = \frac{58}{7}}
\]

\[ %subtraktion #1
\systeme{
	2x  +  4y  -   z  =  17,
	       2y  +  5z  =  19,
		   \frac{38}{7}y  -  \frac{5}{7}z  = \frac{61}{7}}
\]

\[ %auf gleich multiplizieren #2
\systeme{
	2x  +  4y  -   z  =  17,
		   2y  +  5z  =  19,
		   2y  -  \frac{5}{19}z  = \frac{61}{19}}
\]

\[ %subtraktion #2
\systeme{
	2x  +  4y  -   z  =  17,
		2y  +  5z  =  19,
			\frac{100}{19}z  = \frac{300}{19}}
\]

%TODO: von unten nach oben auflösen

Das Verfahren findet man zudem äußerst gut dokumentiert auf der Webseite von Dieter Heidorn\footcite{math_gauss_verfahren}. %Webseite drin lassen???

\subsection{Implementation des Mathematik-Pakets in die GUI}
Als sowohl die GUI als auch das Mathematik-Paket fertiggestellt waren, war per Benutzereingabe immer noch keine Rechnung durchführbar. Was zu diesem Zeitpunkt noch fehlte, war eine passende Implementation, die sicher stellte, dass all unsere festgelegten Strukturen in der Software auch aus Benutzereingaben übersetzt werden konnten. Da dieses Thema letzten Endes sehr viel Aufwand in Anspruch nahm, wollten wir ihm ein eigenes Kapitel widmen.\\
\\
Die zwei wichtigsten Aufgaben beim Kommunizieren zwischen Mathematik-Paket und GUI sind einerseits das Darstellen von \displaycode{MathObject}s und andererseits das Übersetzen in \displaycode{MathObject}s, das wir, wie es in der Informatik üblich sein zu scheint, \textit{parsen} (aus dem Englischen: ''to parse'') nennen.

\subsubsection{Darstellen mathematische Strukturen}
Bezüglich der Darstellung waren an diesem Fall schon Vorgaben aus der GUI hinsichtlich des \displaycode{Sprite}-Interfaces. Jedes \displaycode{MathObject} musste von nun an, da zum anzeigen ja eine Implementierung des Interfaces nötig war, sämtliche Methoden zur Darstellung beinhalten, wie zum Beispiel \displaycode{void paint(Graphics g)} eine war.

\subsection{Arduino und C}

Um das Ein- und Ausschalten der Hardware-Komponenten möglich zu machen wurde der verwendete Arduino Mikrocontroller mithilfe der Arduino IDE programmiert. Der Arduino wird mit der Programmiersprache C programmiert, da ich diese Programmiersprache bereits beherrscht habe, ist mir die Programmierung leicht gefallen.\\
\\
Es sollten folgende Spezifikationen erfüllt werden:
\begin{itemize}
	\item Wenn der Knopf gedrückt und innerhalb von 3 Sekunden losgelassen wird soll, solange der Raspberry Pi eingeschaltet ist, der Displaystrom getoggelt, also je nach Zustand aus- oder eingeschaltet werden.
	\item Wenn der Knopf gedrückt und nach 3-sekündigem Halten losgelassen wird soll, solange der Raspberry Pi eingeschaltet ist, dem Raspberry Pi ein Signal zum Herunterfahren geschickt werden. Sobald dieser Shutdown-Prozess beendet ist sollen Raspberry und Display vom Strom getrennt werden.
	\item Wenn der Knopf gedrückt und nach 3-sekündigem Halten losgelassen wird soll, solange der Raspberry Pi ausgeschaltet ist, der Strom zum Raspberry Pi und zum Display frei gegeben d.h. eingeschaltet werden.
\end{itemize}
\ \\
Um im Standby-Modus (wenn Display und Raspberry Pi ausgeschaltet sind) möglichst stromsparend zu arbeiten, wurde die LowPower-Library \footcite{lowpower_lib} verwendet und mit Interrupts programmiert, dies macht es möglich den Mikrocontroller in einen Schlafmodus zu versetzen, so hat dieser einen sehr geringen Arbeitsstrom.
\\
Dies wurde folgendermaßen realisiert:
\displayownimageg{img/frequem/arduino_loop}{Arduino Event Loop}{Michael Friesenhengst}

\subsection{Python auf dem Raspberry Pi}\displayauthor{Michael Friesenhengst}\ \\
Um das Shutdown-Signal, welches der Arduino sendet, am Raspberry Pi zu verarbeiten, habe ich mich für die Programmiersprache Python entschieden. Python wurde deshalb verwendet, da die Libraries für die Ansteuerung der GPIO-Pins sehr gut dokumentiert und einfach zu verwenden sind.\\
\\
Sobald ein HIGH-Signal auf Pin 23 des Raspberry Pi eingeht, führt dieser einen Befehl zum Herunterfahren aus. Um dem Arduino zu signalisieren, wann der Raspberry fertig heruntergefahren ist, wird Pin 24 standardmäßig auf ein HIGH-Potential gesetzt, nach Beendigung des Shutdown-Vorgangs wird dieser Pin automatisch auf LOW gesetzt.
\\
Folgender Programmcode wurde entwickelt:

\displayownimageg{img/frequem/rpi_python_shutdown}{Raspberry Pi Python Shutdown-Button}{Michael Friesenhengst}
\section{Marketing}
Wenn erste Versionen dieses Produkts funktionstüchtig und einsatzbereit sind, ist allerdings noch nicht alle Arbeit getan. Der nächste Schritt ist für uns herauszufinden, wie wir unser Produkt in die Mengen bringen und es damit marktfähig machen können, und das auch umzusetzen. Da wir selbst noch keinerlei praktische Marketingerfahrung haben, nahmen wir uns einige Marketing-Werkzeuge aus dem Buch ''Marketing - Grundlagen makrtorientierter Unternehmensführung\footcite{book_marketing}'' zur Hand, die sich u.a. die Abgrenzung des \textit{relevanten Marktes}, Käuferverhaltensforschung und Markenanalyse zu Nutze machten.\\

\subsection{Marktabgrenzung}
Beim Thema Marktabgrenzung scheint es verschiedene Herangehensweisen zu geben. Wir haben uns entschieden, zuerst die gängigen Kriterien und anschließend die Abgrenzung, zugeschnitten auf unser Produkt, zu behandeln.\\

\subsubsection{Kriterien zur Marktabgrenzung}
Die Kriterien, um zu bestimmen, welche Kunden in jenes Marktsegment fallen, das wir erreichen möchten, beschränken sich auf drei relativ einfache und doch wichtige Bereiche:\\

\begin{itemize}
	\item Sachlich: Welche Arten von Leistungen werden am Markt angeboten?
	\item Zeitlich: Ist der Markt zeitlich begrenzt?
	\item Räumlich: Ist der Markt lokal, regional, national oder international begrenzt?
\end{itemize}
\ \\

Die Frage um die sachlichen Kriterien würden wir so interpretieren, dass es nicht nur wichtig ist, seine direkte Konkurrenz, auf die ich später noch genauer eingehen möchte, zu betrachten, sondern auch \textit{bedürfnisorientiert} zu denken. In der Lektüre wird hierbei ein Beispiel mit einer Bohrmaschine angeführt, die nicht nur besser sein sollte als andere Bohrmaschinen, sondern auch eine praktische Alternative zu anderen Befestigungsmethoden darstellen soll. Das bedeutet übertragen auf unser Produkt, dass wir begannen, uns Gedanken zu machen, was nötig wäre, um eine volle Alternative zu unserem Produkt zu haben. Um als Kunde tatsächlich eine Alternative zu EPIC zu haben, ist es notwendig, ein Notebook als digitale Mitschreib-Gelegenheit, einen Block für Zeichnungen mit der Hand und einen Taschenrechner oder ein Mathematik-Programm für höhere Rechnungen mit Quer-Referenzen zu besitzen. Dass all dies nötig ist, ist keine Überraschung, da es genau das Grundkonzept von EPIC war, diese Dinge zu vereinen. \\
\\
Zur Frage um die zeitliche Begrenztheit ist die Antwort ganz klar ''Ja''. Unserer Meinung nach ist alles digitale oder elektronische, was für die Masse konzipiert ist, mit einem Ablaufdatum versehen. Ich denke, wir brauchen diese Meinung nicht länger zu begründen, da schon viele feststellten, dass sich die Entwicklung der Technologie schon seit einiger Zeit im exponentiellen Wachstum befinden zu scheint. Was unsere Reaktion darauf sein sollte ist, denke ich, offensichtlich. Alle Unternehmen, die sich auf moderne Technologie spezialisieren müssen am Puls der Zeit bleiben oder sie verlieren ihren Marktanteil. Das wird bei dieser Arbeit ähnlich verlaufen.\\
\\
Nach unserer Meinung kann man die Frage um das Räumlich auf zwei verschiedene Weisen verstehen: ''Ist der Markt im praktischen Sinne, also \underline{zur Zeit} räumlich begrenzt?'' oder ''Ist der Markt theoretisch, also mit \underline{beliebigen Kapazitäten} begrenzt?''\\
Da unsere Kapazitäten nicht beliebig wählbar sind, denke ich, macht es wenig Sinn, daran viele Gedanken zu verschwenden, aber generell kann man den Markt in etwa auf die westliche Welt einschränken, wobei möglicherweise noch Australien und Industriestaaten im Osten in Frage kämen.\\
Nun aber zum realistischen Teil, nämlich was wir mit unserer aktuellen Marketing-Stärke erreichen können. Niederösterreich und eventuell Wien, da die Zielgruppe, wie es später noch genauer definiert wird, hauptsächlich Studierende sein sollen, sind für uns realitätsnahe Ziele. Weiters könnten auch Kunden anderer Universitätsstädten bei uns kaufen. Niederösterreich und Wien sollen jedoch vorerst unsere Hauptziele bleiben.\\

\subsubsection{relevanter Markt}
Unter dieser Überschrift zielen wir auf eine noch detailliertere Abgrenzung ab. Mit weiteren Fragen zur Einschließung wollen wir unsere Zielgruppe konkretisieren.\\
Es stellt sich heraus, dass hier insgesamt drei für uns interessante Fragen auftreten. Andere Vorschläge für diesen Bereich sind unserem Urteil nach für Jungunternehmen nicht relevant.\\

\begin{itemize}
	\item Wie viele Nachfrager beinhaltet der Markt?
	\item Zeitlich: Wie viele Anbieter beinhaltet der Markt, und welche Anbieter gehören zu den Hauptkonkurrenten?
	\item Hat ein Unternehmen eine marktbeherrschende Stellung, sodass wettbewerbsrechtliche Vorgaben nicht mehr eingehalten werden?
\end{itemize}
\ \\
Die Nachfrager im Markt umfassen alle Studierenden, wobei der Fokus wahrscheinlich auf technischen Studienrichtungen liegt, Schüler aus höheren Schulen und möglicherweise Techniker, die viel im theoretischen Bereich zu tun haben.\\
\\
Wie bereits erwähnt sehen wir zu unserem Produkt keine direkte Konkurrenz, da zwar sowohl Taschenrechner als auch etwa ''digitale Notzblöcke'' am Markt erhältlich sind, aber nichts, was beides zusammenfasst. Aus diesem Grund möchten wir eine geteilte Konkurrenz im weiteren Sinne betrachten. Auf der Seite der Taschenrechner und Rechenprogrammen ist uns beispielsweise Mathcad wahrscheinlich am ähnlichsten, da sowohl Dokumentation als auch Speicherbarkeit und Rechenfunktionalität gegeben sind. Dafür mangelt es bei diesem Produkt zum Vergleich an Portabilität und der Möglichkeit, Freihandzuzeichnen. Als ähnliche Art der Konkurrenz könnte man auch MatLab und eventuell noch GeoGebra sehen. Auf der klassischen Taschenrechner-Seite stehen wir ganz klar TexasInstruments und zu geringerem Anteil auch Casio gegenüber. Betrachtet man hingegen rein den Aspekt der Handmitschrift, so ergibt sich als relativ moderner Konkurrent OneNote, eine Notizen-Applikation von Microsoft.\\
\\
Für eine marktbeherrschende Stellung, sodass wettbewerbsrechtliche Vorgaben nicht mehr eingehalten werden, kämen hier einerseits TexasInstruments und andererseits Microsoft in Frage. Man könnte an dieser Stelle leicht sagen, dass wir zum Glück für Microsoft keine so direkten Konkurrenten wie für beispielsweise Mathcad sind, da deren Wirtschaftsmacht die von TexasInstruments sicher bei weitem übersteigt.\\

\subsection{Aus Sicht des Kunden}
Damit wäre unser typischer Kunde nun klassifiziert. Uns ist bewusst, wen wir in etwa ansprechen, jedoch ist bis jetzt nicht geklärt wie das erfolgt. Aus diesem Grund soll in diesem Abschnitt analysiert werden, wie ein potentieller Kunde denkt und entscheidet, damit wir unser Marketing genau darauf abstimmen können.

\subsubsection{Käuferverhalten}
Im ersten Ansatz bedienten wir uns der Käuferverhaltensforschung. Diese fasst folgende wesentliche Paradigmen zusammen:\\

\begin{itemize}
	\item \tab{Wer kauft?}        \tab{$\rightarrow$ Kaufakteure, Träger der Kaufentscheidung}
	\item \tab{Was?}              \tab{$\rightarrow$ Kaufobjekte}
	\item \tab{Warum?}            \tab{$\rightarrow$ Kaufmotive}
	\item \tab{Wie?}              \tab{$\rightarrow$ Kaufentscheidungsprozess}
	\item \tab{Wie viel?}         \tab{$\rightarrow$ Kaufmenge}
	\item \tab{Wann?}             \tab{$\rightarrow$ Kaufzeitpunkt, -häufigkeit}
	\item \tab{Wo bzw. bei wem?}  \tab{$\rightarrow$ Einkaufsstätten, Lieferantenwahl}
\end{itemize}
\ \\
Da dieses Unterkapitel auf dem letzten aufbauen soll, möchte ich die Frage ''Wer kauft?'' überspringen und auf die Marktabgrenzung verweisen.\\
\\
Was gekauft wird, ist sicher auch nicht schwer zu beantworten, zu unserem Produkt gibt es keine aktuell vorgesehenen Zusatzprodukte, der Kauf beschränkt sich also einzig und allein auf das Gerät EPIC.\\
\\
Die Frage nach dem Warum ist schon interessanter. Verschiedene Unternehmen gewinnen Kunden mit verschiedenen Merkmalen. So kaufen Apple-Kunden vielleicht wegen Design oder Benutzerfreundlichkeit in der Software, und Microsoft-Kunden mögen wegen Langzeit-Support ihrer Produkte oder Kompatibilität bedingt durch die allgemeine Verbreitung der Produkte kaufen. Bei EPIC wollen wir unsere Kunden mit intuitiver Funktionalität überzeugen. Das Ziel ist es, dass Kunden keine Bedienungsanleitungen lesen und nicht mehr als einmal im Web nach einer Anleitung suchen müssen. Das sollte einer der Gründe sein, warum sich Kunden für EPIC entscheiden, gestützt von der Tatsache, dass zur Zeit, wie erwähnt, keine direkten Alternativen am Markt bestehen.\\
\\
Ich denke, der Kaufentscheidungsprozess ist für ein so junges Produkt, das noch nicht am Markt ist, schwer zu definieren, da er sich laufend ändern wird. Am Beginn wird sich sehr viel davon im Bereich der Mundpropaganda abspielen, bis später im Idealfall Werbung auf Plakate oder gar in den Medien finanziert werden könnte.\\
\\
Wie es am Technologie-Markt sicher üblich ist, wird nicht mehr als ein Exemplar pro Kopf benötigt.\\
\\
Von Kaufhäufigkeit kann hier, denke ich, nicht gesprochen werden. Je nach dem, wie sich die Technologie entwickelt, könnte es der Fortschritt alle 1-4 Jahre verlangen, ein neues Gerät anzuschaffen. Über den Kaufzeitpunkt kann womöglich auch nur spekuliert werden, aber möglicherweise zeichnet sich ein Muster ab, dass im Herbst vorrangig gekauft wird, da in dieser Jahreszeit Schule und Universität beginnen.\\
\\
EPIC soll in jedem Fall im Internet erreichbar sein, womit sich die Frage nach dem Wo weitgehend klärt. Natürlich wird es auch möglich sein, Käufe über persönliches In-Kontakt-Treten abzuwickeln.\\

\subsubsection{Produktnutzen}
Die Analyse des Kaufverhaltens hat einige interessante Ergebnisse gebracht. Was allerdings noch ein wenig klarer werden könnte, ist das Kaufmotiv, weshalb wir hier den Produktnutzen in den Fokus nehmen wollen. Durch die Definition des Produktnutzens bekommen wir einen noch tieferen Einblick, warum man sich entscheiden könnte, EPIC zu kaufen.\\
\\
Eine sinnvolle Einteilung und Aufspaltung dessen ist die in \textit{Grundnutzen} und \textit{Zusatznutzen}, der sich wiederum in \textit{Erbauungsnutzen} und \textit{Geltungsnutzen} teilt. Genauer versteht man unter Grundnutzen \citebrackets{die aus den technisch-funktionalen Basiseigenschaften eines Produktes resultierende Bedürfnisbefriedigung}, die in unserem Fall eindeutig ganz simpel durch die praktische Funktionalität gegeben ist.\\
\\
Der Zusatznutzen wird eben in die \citebrackets{aus den ästhetischen Wirkungen eines Produktes} (Erbauungsnutzen) und die \citebrackets{aus den sozialen Wirkungen eines Produktes} (Geltungsnutzen) resultierende Bedürfnisbefriedigung unterteilt. Im konkreten Fall ist vielleicht mit gutem Erscheinungsbild bei einem Taschenrechner nicht viel zu holen, da er sicher nicht gekauft würde, wenn er wenig Funktionalität, dafür gutes Design aufweisen würde. Dafür kann sicher zu einem gewissen Grad der Negativ-Zufriedenheit entgegengesteuert werden; Getreu der Denkweise: ''Schlechtes Design könnte Kunden am Kaufen hindern, obwohl gutes Design den Kauf wahrscheinlich nicht fördert.'' Daraus leiten wir ab, dass die Benutzeroberfläche ansprechend gestaltet sein sollte und das Gerät eine seriöse Arbeitsweise verkörpert. Den Geltungsnutzen hingegen sehen wir bei EPIC sehr beschränkt, da sich keinerlei soziale Vorteile in einer Gruppe durch den Besitz eines Taschenrechners ergeben.\\

\subsubsection{Preisabhängige Qualitätsbeurteilung}
Als letztes Analysewerkzeug nahmen wir die Feststellung des empfundenen Kaufrisikos heran. Dieses Tool ist sehr stark mit dem Kaufentscheidungsprozess verknüpft. Entscheidet sich ein Kunde erstmalig, ein bestimmtes Produkt zu kaufen, so fand davor der Kaufentscheidungsprozess statt, mithilfe dessen er in den letzten Phasen seine präferierte Entscheidung mit Alternativen vergleicht. Als eine Art Abkürzung kann man es nun sehen, wenn das Kaufrisiko gesenkt werden kann. Beispielsweise würde ein Käufer bei einem preislich überdurchschnittlichen Mobiltelefon länger überlegen und sich fragen, ob es sich nun wirklich auszahlt, als bei einem mit durchschnittlicher Preisgestaltung. Wie man allerdings gerade am Beispiel der Mobiltelefone erkennen kann, ist es möglich dieses \textit{empfundene Kaufrisiko} auf andere Arten als durch Änderung der Preispolitik zu senken. Neben den offensichtlichen Faktoren wie Qualitätsbestreben und Zeitdruck gibt es noch viele mehr, wie in der folgenden Grafik zu sehen ist.\\
%\textbf{IMAGEIMAGEIMAGEIMAGEIMAGEIMAGEIMAGEIMAGE}\\
\displayimageg{img/sachsenschnitzel/book_Marketing_Kaufrisiko}{Einflussfaktoren für preisorientierte Qualitätsbeurteilung und das Kaufrisiko\footcite{book_marketing}} %Abb. 4-48 S.489 Kap. 4: 2.327

\subsection{Umsetzung}
Nach dem Einsatz all dieser Marketing-Werkzeuge ist uns nun klar, welche Kunden wir ansprechen, welchen ihrer Kaufmotive wir treu werden sollen und was ihre Entscheidung beeinflusst. Der nächste Schritt ist jetzt logischerweise, das gesammelte Wissen zu beweisen. Zusammengefasst sind unsere Ziele abgeleitet aus den Analyseergebnissen für den ersten Kundenkontakt einerseits, unsere Zielgruppe, Schüler oder Studenten nicht Technik-ferner Fachrichtungen anzusprechen, und andererseits, auf Funktionalität und Praktikabilität hinzuweisen und zu betonen. Der optimale Weg, das zu erreichen war in unseren Augen, eine Präsentation von EPIC an der Schule abzuhalten.
\section{Danksagung}
Wir möchten in erster Linie Herrn Ing. Leopold Mayer für seine Unterstützung danken. Er begleitete uns bei unserer Arbeit und stellte facheinschlägige Literatur zur Verfügung.\\
Außerdem möchten wir uns bei Herrn Ing. Thomas Panzenböck für seine Unterstützung hinsichtlich der Benutzung des 3D-Druckers bedanken.\\
Im Übrigen möchten wir all jenen Dank aussprechen, die zur Ideenfindung mathematischer Lösungsansätze beitrugen.
\section{Zeitaufzeichnung}
\subsection{Michael Friesenhengst}
\begin{tabular}{l | l | r}
	Datum & Tätigkeit & Dauer\\\hline
	\formatdate{28}{09}{2016} & Besprechung: Grundstruktur und grobe Durchführung & 2:00\\\hline
	
	\formatdate{01}{10}{2016} & Überlegung von Strukturen & 4:00\\\hline
	\formatdate{08}{10}{2016} & Suchen von Hardware-Komponenten & 4:30\\\hline
	\formatdate{15}{10}{2016} & Tests in Java & 3:00\\\hline
	\formatdate{23}{10}{2016} & Erster Zusammenbau der Hardware und Tests & 5:00\\\hline
	\formatdate{25}{10}{2016} & Besprechung: Hardware-Komponenten & 2:00\\\hline
	\formatdate{29}{10}{2016} & Installation von nötigen Paketen auf dem Raspberry Pi & 2:00\\\hline
	
	
\end{tabular}
\subsection{Florian Weinzerl}
\begin{longtabu} to \textwidth{l | X | r}
	\textbf{Datum} & \textbf{Tätigkeit} & \textbf{Dauer}\\\hline
	
	% September
	\formatdate{28}{09}{2016} & Besprechung: Grundstruktur und grobe Durchführung & 2:00\\\hline
	
	% Oktober
	\formatdate{6}{10}{2016} & erste Software-Tests mit Termen & 2:00\\\hline
	\formatdate{7}{10}{2016} & erste Software-Tests mit Termen & 3:00\\\hline
	\formatdate{9}{10}{2016} & Auslegung der Term-Struktur & 4:00\\\hline
	\formatdate{10}{10}{2016} & Auslegung der Term-Struktur & 5:00\\\hline
	\formatdate{18}{10}{2016} & Bug-fixes & 1:00\\\hline
	\formatdate{19}{10}{2016} & Bug-fixes & 0:30\\\hline
	\formatdate{22}{10}{2016} & hinzufügen von Fraction und Product & 2:00\\\hline
	\formatdate{25}{10}{2016} & Besprechung: Hardwarekomponenten & 2:00\\\hline
	
	% November
	\formatdate{8}{11}{2016} & Gleichungssystem Lösungen recherchieren & 2:00\\\hline
	\formatdate{9}{11}{2016} & Gleichungen in Software implementieren & 2:00\\\hline
	\formatdate{13}{11}{2016} & Support für Gleichungssysteme in Software & 2:00\\\hline
	\formatdate{14}{11}{2016} & Bug fixes & 3:00\\\hline
	\formatdate{15}{11}{2016} & Term-Struktur für Gleichungen und Funktionen angepasst & 5:00\\\hline
	\formatdate{18}{11}{2016} & Zusammenführen der bisherigen Mathematik-Komponenten & 3:00\\\hline
	\formatdate{21}{11}{2016} & Bug fixes & 3:00\\\hline
	\formatdate{22}{11}{2016} & Plot-Methode für Funktionen implementiert & 3:00\\\hline
	\formatdate{23}{11}{2016} & Plot-Methode für Funktionen implementiert & 3:00\\\hline
	\formatdate{26}{11}{2016} & Framework für Funktionen in Term-Struktur zusammengestellt & 2:00\\\hline
	\formatdate{27}{11}{2016} & Framework für Funktionen in Term-Struktur zusammengestellt & 2:00\\\hline
	\formatdate{28}{11}{2016} & Recherchieren mathematischer Lösungswege für Funktionen  & 1:00\\\hline
	
	% Dezember
	\formatdate{4}{12}{2016} & mathematische Lösungen recherchieren & 1:00\\\hline
	\formatdate{5}{12}{2016} & Strukturierung des Nullstellen-Programms & 3:00\\\hline
	\formatdate{6}{12}{2016} & Strukturierung des Nullstellen-Programms & 2:00\\\hline
	\formatdate{9}{12}{2016} & mathematische Lösungen recherchieren & 2:00\\\hline
	\formatdate{10}{12}{2016} & mathematische Lösungen recherchieren & 2:00\\\hline
	\formatdate{12}{12}{2016} & Implementierung von Funktionen in die Software & 2:00\\\hline
	\formatdate{13}{12}{2016} & Besprechung: erstes GUI-Review & 2:00\\\hline
	\formatdate{14}{12}{2016} & Implementierung von Funktionen in die Software & 4:00\\\hline
	\formatdate{15}{12}{2016} & Implementierung von Funktionen in die Software & 1:00\\\hline
	
	% Jänner
	\formatdate{10}{01}{2017} & verallgemeinerung von \displaycode{Term} als \displaycode{MathObject} & 3:00\\\hline
	\formatdate{11}{01}{2017} & Besprechung: Zwischenstand & 1:00\\\hline
	\formatdate{13}{01}{2017} & Dokumentation der Mathematik-Implementierungen & 2:00\\\hline
	\formatdate{14}{01}{2017} & Dokumentation der Mathematik-Implementierungen & 2:00\\\hline
	\formatdate{15}{01}{2017} & Dokumentation der Mathematik-Implementierungen & 1:00\\\hline
	\formatdate{16}{01}{2017} & Dokumentation der Mathematik-Implementierungen & 1:00\\\hline
	\formatdate{21}{01}{2017} & Hinzufügen von Klammer-Termen in Software & 2:00\\\hline
	\formatdate{22}{01}{2017} & Hinzufügen von Klammer-Termen in Software & 3:00\\\hline
	\formatdate{24}{01}{2017} & Bug fixes & 2:30\\\hline
	\formatdate{25}{01}{2017} & fertigstellen und synchronisieren erster Version mit Mathematik-Paket & 4:30\\\hline
	\formatdate{29}{01}{2017} & Darstellung mathematischer Komponenten implementiert & 2:00\\\hline
	\formatdate{30}{01}{2017} & Darstellung mathematischer Komponenten implementiert & 2:00\\\hline
	\formatdate{31}{01}{2017} & Darstellung mathematischer Komponenten implementiert & 3:00\\\hline
	
	% Feber
	\formatdate{2}{02}{2017} & Zusammenfassung Dokumentation der Arbeitszeiten & 1:00\\\hline
	\formatdate{6}{02}{2017} & Darstellung mathematischer Komponenten implementiert & 2:00\\\hline
	\formatdate{7}{02}{2017} & Bug fixes & 2:00\\\hline
	\formatdate{8}{02}{2017} & Cursor-Support für \displaycode{MathObjects} & 2:00\\\hline
	\formatdate{9}{02}{2017} & Cursor-Support für \displaycode{MathObjects} & 2:00\\\hline
	\formatdate{11}{02}{2017} & Cursor-Support für \displaycode{MathObjects} & 2:00\\\hline
	\formatdate{13}{02}{2017} & Zusammenführen von Mathematik- und GUI-Komponenten & 2:00\\\hline
	\formatdate{14}{02}{2017} & geringfügige Änderungen & 1:00\\\hline
	\formatdate{24}{02}{2017} & Bug fixes & 3:00\\\hline
	
	% März
	\formatdate{7}{03}{2017} & Auslegung der Parsing-Struktur & 1:00\\\hline
	\formatdate{8}{03}{2017} & Auslegung der Parsing-Struktur & 3:00\\\hline
	\formatdate{9}{03}{2017} & Parsing-Test in Java & 3:00\\\hline
	\formatdate{11}{03}{2017} & Parsing-Support in Software & 2:00\\\hline
	\formatdate{12}{03}{2017} & Parsing-Support in Software & 4:00\\\hline
	\formatdate{13}{03}{2017} & Parsing-Support in Software & 2:00\\\hline
	\formatdate{16}{03}{2017} &  & 1:00\\\hline
	\formatdate{17}{03}{2017} & allfällige Tests der Term-Struktur-Neuimplementationen & 4:00\\\hline
	\formatdate{18}{03}{2017} & Fertigstellung der Mathe-GUI-Implementation & 9:00\\\hline
	\formatdate{19}{03}{2017} & Fertigstellung der Mathe-GUI-Implementation & 6:00\\\hline
	\formatdate{20}{03}{2017} & Bug fixes \& geringfügige Änderungen & 3:00\\\hline
	\formatdate{21}{03}{2017} & Marketing & 4:00\\\hline
	\formatdate{22}{03}{2017} & Besprechung: Fertigstellung der Software und Hardware, Marketingkonzepte & 3:00\\\hline
	\formatdate{23}{03}{2017} & Marketing & 1:00\\\hline
	\formatdate{25}{03}{2017} & Dokumentation & 3:00\\\hline
	\formatdate{26}{03}{2017} & Dokumentation & 3:00\\\hline
	\formatdate{27}{03}{2017} & Dokumentation & 2:00\\\hline
	\formatdate{28}{03}{2017} & Dokumentation & 2:30\\\hline
	\formatdate{29}{03}{2017} & Dokumentation & 2:00\\\hline
	\formatdate{30}{03}{2017} & Dokumentation & 3:00\\\hline
	\formatdate{31}{03}{2017} & Dokumentation & 5:00\\\hline
	
	% April
	\formatdate{1}{04}{2017} & Dokumentation & 6:00\\\hline
	\formatdate{2}{04}{2017} & Dokumentation & 6:00\\\hline
	\formatdate{3}{04}{2017} & Fertigstellung der Arbeit & 4:00\\\hline
	\formatdate{04}{04}{2017} & Fertigstellung und Abgabe der Arbeit & 4:00\\\hline
	\multicolumn{2}{r|}{\textbf{Summe}} & 202:00
\end{longtabu}

\listoffigures
\newpage
\printbibliography

\end{document}